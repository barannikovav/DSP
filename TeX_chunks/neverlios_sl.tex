
\section{n-мерная плотность конечно мерного распределения случайного процесса}

\textbf{Автор:} Паншин Артём Владимирович, Б-01-001

\subsection{n-мерная плотность распределения, определение и примеры [3]} % should be \cite{ShiryaevVeroyatnost1980}
Большой запас $n$-мерных функций распределения имеет вид
$$
F_n\left(x_1, \ldots, x_n\right)=\int_{-\infty}^{x_1} \ldots \int_{-\infty}^{x_n} f_n\left(t_1, \ldots, t_n\right) d t_1 \ldots d t_n,
$$
где $f_n\left(t_1, \ldots, t_n\right)$ - неотрицательные функции $\mathrm{c}$
$$
\int_{-\infty}^{\infty} \ldots \int_{-\infty}^{\infty} f_n\left(t_1, \ldots, t_n\right) d t_1 \ldots d t_n=1
$$
а интегралы понимаются в смысле Римана (и в более общем случае в смысле Лебега). Функции $f=f_n\left(t_1, \ldots, t_n\right)$ называют плотностями n-мерной функции распределения, п-мерной плотностью распределения вероятностей или просто п-мерными плотностями.
В случае $n=1$ функция
$$
f(x)=\frac{1}{\sqrt{2 \pi} \sigma} e^{-\frac{(x-m)^2}{2 \sigma^2}}, \quad x \in R
$$
c $\sigma>0$ есть плотность (невырожденного) гауссовского, или нормального, распределения. Существуют естественные аналоги этой плотности и в случае $n>1$.

Пусть $\mathbb{R}=\left| r_{i j}\right|$ - некоторая неотрицательно определенная симметрическая матрица порядка $n \times n$ :
$$
\begin{gathered}
\sum_{i, j=1}^n r_{i j} \lambda_i \lambda_j \geqslant 0, \quad \lambda_i \in R, \quad i=1, \ldots, n, \\
r_{i j}=r_{j i} .
\end{gathered}
$$
В том случае, когда $\mathbb{R}$ - положительно определенная матрица, ее детерминант $|\mathbb{R}| \equiv \operatorname{det} \mathbb{R}>0$, и, следовательно, определена обратная матрица $A=\left|a_{i j}\right|$. Тогда функция
$$
f_n\left(x_1, \ldots, x_n\right)=\frac{|A|^{1 / 2}}{(2 \pi)^{n / 2}} \exp \left\{-\frac{1}{2} \sum a_{i j}\left(x_i-m_i\right)\left(x_j-m_j\right)\right\},
$$
где $m_i \in R, i=1, \ldots, n$, обладает тем свойством, что интеграл (Римана) от нее по всему пространству равен 1, и, следовательно, в силу ее положительности она является плотностью.

Эта функция называется плотностью n-мерного (невырожденного) гауссовского, или нормального, распределения (с вектором средних значений $m=\left(m_1, \ldots, m_n\right)$ и матрицей ковариаций $\left.\mathbb{R}=A^{-1}\right)$.
В случае $n=2$ плотность $f_2\left(x_1, x_2\right)$ может быть приведена к виду
$$
 f\left(x_1, x_2\right)=\frac{1}{2 \pi \sigma_1 \sigma_2 \sqrt{1-\rho^2}} \exp \left\{-\frac{1}{2\left(1-\rho^2\right)} \times\right. \\
$$
$$
\left.\times\left[\frac{\left(x_1-m_1\right)^2}{\sigma_1^2}-2 \rho \frac{\left(x_1-m_1\right)\left(x_2-m_2\right)}{\sigma_1 \sigma_2}+\frac{\left(x_2-m_2\right)^2}{\sigma_2^2}\right]\right\} ,
$$
где $\sigma_i>0,|\rho|<1$.

\begin{remark} Как и в случае $n=1$, теорема 2 допускает обобщение на (аналогичным образом определяемые) меры Лебега-Стилтьеса в $\left(R^n, \mathscr{B}\left(R^n\right)\right)$ и обобщенные функции распределения в $R^n$. В том случае, когда обобщенная функция распределения $G_n\left(x_1, \ldots, x_n\right)$ равна $x_1 \ldots x_n$, соответствующая мера называется мерой Лебега на борелевских множествах пространства $R^n$. Ясно, что для нее
$$
\lambda(a, b]=\prod_{i=1}^n\left(b_i-a_i\right),
$$
т. е. мера Лебега «прямоугольника»
$$
(a, b]=\left(a_1, b_1\right] \times \ldots \times\left(a_n, b_n\right]
$$
\end{remark}

\subsection{Основные свойства $n$-мерных плотностей вероятности [8]} % should be  \cite{MaltsevBTRPR2014}
Плотности вероятности случайного процесса должны удовлетворять следующим условиям:

1. \textbf{Неотрицательность}: $f\left(x_1, t_1 ; x_2, t_2 ; \ldots ; x_n, t_n\right) \geq 0$;

2. \textbf{Нормировка}: $\int_{-\infty}^{\infty} \cdots \int_{-\infty}^{\infty} f\left(x_1, t_1 ; x_2, t_2 ; \ldots ; x_n, t_n\right) d x_1 d x_2 \ldots d x_n=1$;

3. \textbf{Симметрия}: $f\left(x_1, t_1 ; x_2, t_2 ; \ldots ; x_n, t_n\right)=f\left(x_2, t_2 ; x_1, t_1 ; \ldots ; x_n, t_n\right).$
Это свойство означает, что плотности вероятности симметричны относительно любых перестановок пар аргументов $\left(x_i, t_i\right), i=1,2, \ldots, n$. Например, для двумерной плотности вероятности $f\left(x_1, t_1 ; x_2, t_2\right) \equiv f\left(x_2, t_2 ; x_1, t_1\right)$. Это условие становится очевидным, если учесть, что содержание рассматриваемого события - совместное осуществление двух (или «n») неравенств $P\left\{\begin{array}{c}x_1 \leq x\left(t_1\right)<x_1+d x_1 \\ x_2 \leq x\left(t_2\right)<x_2+d x_2\end{array}\right\}$ не зависит от того, в каком порядке эти неравенства записываются:
$$
f\left(x_1, t_1 ; x_2, t_2\right) d x_1 d x_2 \equiv f\left(x_2, t_2 ; x_1, t_1\right) d x_1 d x_2 \equiv 
$$
$$ 
\equiv P\left\{X_i \leq X\left(t_i\right)<X_i+d X_i\right\}, \quad i=1,2
$$

4. \textbf{Согласованность.}
Данное свойство означает, что из плотности вероятности большей размерности всегда можно получить плотность вероятности меньшей размерности путем интегрирования по «лишним» аргументам. При любом $m<n$ имеем
$$f\left(x_1, t_1\right. ; \left.x_2, t_2 ; \ldots ; x_n, t_n\right)=\int_{-\infty}^{\infty} \cdots \int_{-\infty}^{\infty} f\left(x_1, t_1 ; x_2, t_2 ; \ldots ; x_n, t_n\right) d x_{m+1}$$$$ d x_{m+2} \ldots d x_n$$
Условие согласованности (как и предыдущие условия 1-3) в теории случайных процессов не является тривиальным повторением аналогичного соотношения для плотностей вероятности совокупности случайных величин. Оно ограничивает класс допустимых функций $f\left(x_1, t_1\right.$; $\left.x_2, t_2 ; \ldots ; x_n, t_n\right)$, поскольку интегрирование по $x_{m+1}, \ldots, x_n$ должно автоматически приводить к результату, не зависящему от параметров $t_{m+1}, t_{m+2}, \ldots, t_n$.

Перечисленные четыре свойства должны выполняться для плотности вероятности любого случайного процесса.
