
% for numerated formulas
\newcommand{\formula}[2]
{
    \begin{equation}\label{#1}
        #2
    \end{equation}
}

\newcommand{\defHead}[2]
{
    {\bf\large Определение: #1} \\ $ [ $ Источник: #2 $ ] $ \\
}

\newcommand{\thHead}[2]
{
    {\bf\large Теорема: #1} \\ $ [ $ Источник: #2 $ ] $ \\
}

\newcommand{\refShamarovLecture}[1]
{
    Лекции Н.Н. Шамарова по ДСП в МФТИ, 2023. Лекция \textnumero #1
}

\newcommand{\refShiryaevBook}[1]
{
    А.Н. Ширяев, \guillemotleft Вероятность - 1 \guillemotright, стр. #1.
}

\newcounter{PicsCounter}
\setcounter{PicsCounter}{1}

\newcommand{\pic}[3]{
    \begin{center}
        \begin{minipage}[h!]{#1}
            \begin{center}

                \includegraphics[width = \textwidth]{#2}
                \textit{Рис \arabic{PicsCounter}. #3}

            \end{center}
        \end{minipage}
    \end{center}

    \stepcounter{PicsCounter}
}

\newcounter{TablesCounter}
\setcounter{TablesCounter}{1}

\newcommand{\tableLable}[1]{
    \textit{Таблица \arabic{TablesCounter}: #1}

    \stepcounter{TablesCounter}
}

\newcommand{\defeq}{\stackrel{\text{def}}{=}}

\section{Представление мер произвольного экспериментального распределения (многомерного) с помощью мер Дирака}

\textbf{Автор:} Матренин Василий Николаевич, Б-01-008

\subsection{Необходимые определения}

\defHead{Простая случайная величина}{\refShiryaevBook{54}}
Всякая числовая функция $ \xi = \xi \left(\omega\right) $, определенная на (конечном) пространстве
элементарных событий $ \Omega $, будет называться \textit{(простой) случайной величиной}. \\ [0.2cm]

\defHead{Экспериментальный ансамбль}{\refShamarovLecture{1}}
Совокупность элементарных экспериментов, в ходе которых случайная величина принимает конкретные значения,
называется \textit{экспериментальным ансамблем}. \\ [0.2cm]

\begin{table}[h!]
    \begin{center}

        \tableLable{Экспериментальный ансамбль}
        \begin{tabular}{|l|c|c|c|>{\centering}m{3.5cm}|c|}
        \hline
        $N_{\text{элем}}$ & 1     & 2     & 3     & \dots & $N_{\text{Э}}$       \\ \hline
        $\xi$             & $x_1$ & $x_2$ & $x_3$ & \dots & $x_{N_{\text{Э}}}$   \\ \hline
        \end{tabular}

    \end{center}
\end{table}


\defHead{Интегральная функция распределения}{\refShamarovLecture{1}}
Числовая функция вещественной переменной:

\[
F_{\xi, \text{Э}} \left(c\right) = P \left(\xi \leqslant c\right) =
\frac{\text{\small Кол-во элементарных экспериментов, в которых} \, \xi \leqslant c}
     {\text{\small Общее кол-во элем. эксп-ов в ансамбле, отвечающем эксп-у Э}} =
\frac{|\left\{x_n : x_n \leqslant c|\right\}}{N_{\text{Э}}}
\]

Называется \textit{интегральной функцией распределения}. \\

Свойства $ F_{\xi, \text{Э}} \left(c\right) $:
\begin{enumerate}
    \item Неотрицательность
    \item Принимает конечное число значений, $ 0 \leqslant F \left(c\right) \leqslant 1 $
    \item Монотонно возрастает
    \item $ \lim_{x\to -\infty} F \left(c\right) = 0 $
    \item $ \lim_{x\to +\infty} F \left(c\right) = 1 $

    %%% Copied from: https://www.overleaf.com/project/63e73389cd80241abecca9c8
    В Колмогоровской версии, доказательство будет выглядеть так:\\
    Из определения интегральной функции распределения следует, что равенство F(x) = P(X < x) равносильно $F(x) = P(-\infty < X < x)$. Поэтому 
    \[\lim_{x \rightarrow + \infty} = \lim_{x \rightarrow + \infty} P(X < x) = \lim_{x \rightarrow + \infty} P(-\infty < X < + \infty) = 1 \]
    Так как событие $(-\infty < X < +\infty)$ состоит в том, что случайная величина Х в результате исхода испытания примет какое-то действительное число, является событием достоверным.\\
    ч.т.д
    %%%

\newpage

    \item Имеет одностороннюю непрерывность справа

    %%% Copied from: https://www.overleaf.com/project/63e73389cd80241abecca9c8
    %%% with little changes
    В Колмогоровской версии, доказательство будет выглядеть так:\\
    Функция распределения непрерывна слева: $\lim_{x \rightarrow a - 0} F (x) = F (a) $\\
    Пусть $a_n$ - возрастающая последовательность и $\lim_{n \rightarrow \infty} a_n = a$. Определим событие $A = \{\xi < a\}$, и события $A_n = \{x < a_n\}$. Очевидно, что для последовательности событий $A_n$ справедливо:
    \[A_1 \subset A_2 \subset ... \subset A_n \subset ... \text{ и} \bigcup_{n=1}^\infty A_n = A\]
    С учётом монотонности функции распределения:
    \[\lim_{x \rightarrow +\infty} F (x) = \lim_{n \rightarrow \infty} P(A_n) = P(A) = F (a)\]
    ч.т.д.\\
    %%%

\end{enumerate}

\defHead{Интегральная функция совместного распределения упорядоченных вещественных случайных величин}
{\refShamarovLecture{1}}
Аналогично можно ввести интегральную функцию совместного распределения упорядоченных вещественных случайных
величин. Например, для пары величин $ \left(\xi_1, \xi_2\right) $ : \\ [0.2cm]
\[
F_{\xi_1, \xi_2, \text{Э}} \left(c_1, c_2\right) = P \left(\xi_1 \leqslant c_1, \xi_2 \leqslant c_2\right) =
\frac{|\left\{ n | x_{1n} \leqslant c_1, x_{2n} \leqslant c_2 \right\}|}{N_{\text{Э}}}
\]
