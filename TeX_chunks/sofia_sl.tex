\section{Среднее время возвращения марковской цепи в существенное состояние}

\textbf{Автор:} Софья Бочкарёва, Б-01-005

\subsection{Основные понятия} 
\textbf{Марковским процессом} называется случайный процесс $\xi (t)$, если его условная плотность распределения не зависит от значений процесса в моменты $t_1, t_2, ..., t_{n-1}$ а определяется лишь значением $\xi(t_n) = x_n$, т.е.

$$p(x_{n + 1}, t_{n + 1} | x_1, t_1; x_2, t_2;...;x_n, t_n) = p(x_{n + 1}, t_{n + 1} | x_n, t_n)$$

\textbf{Марковской цепью} называют марковский процесс, для которого множество $X = \{i_1, i_2, ..., i_n, ...\}$ счетно или конечно. 
\par\medskip
Рассмотрим цепь Маркова с дискретным временем и, для простоты, с конечным числом состояний $X = \{1, 2, ... N\}$. Вероятности

$$p_{ij}^{(n)} = P(\xi_{m+n} = j | \xi_m = i), n \geq 1, m \geq 1, i,j \in X $$

называются \textit{вероятностями перехода цепи Маркова за n шагов}, а $p_{ij}^{(1)} = p_{ij}$ просто \textit{вероятностями перехода}. Матрица вида

\begin{equation*}
	P = \left(
	\begin{array}{cccc}
	p_{11} & p_{12} & \ldots & p_{1N}\\
	p_{21} & p_{22} & \ldots & p_{2N}\\
	\vdots & \vdots & \ddots & \vdots\\
	p_{N1} & p_{N2} & \ldots & p_{NN}
	\end{array}
	\right)
\end{equation*}

называется \textit{матрицей вероятностей перехода цепи Маркова}.
\par\medskip
\textbf{Однородной} называется цепь Маркова, у которой вероятности перехода не зависят от m, т.е. матрица $P$ не зависит от шага m. Тогда очевидно, что свойства однородной марковской цепи зависят только от матрицы $P$ и начального распределения
\par\medskip
\textbf{Несущественным состоянием $i \in X$} называется состояние, из которого можно за положительное число шагов выйти: $\exists m, j : p_{ij}^{(m)} > 0$, но нельзя в него вернуться: \mbox{$\forall n : p_{ij}^{(n)} = 0$.}
\par\medskip
Если из множества $X$ всех состояний выделить несущественные, то оставшееся множество \textbf{существенных} состояний обладает тем свойством, что, попав в него, цепь Маркова никогда из него не выйдет. 

\par\medskip
\textbf{Сообщающимися состояниями} i и j называются существенные состояния, если i достижимо из j, и j достижимо из i. Обозначается как $i \leftrightarrow j$.
\par\medskip
Множество существенных состояний разбивается на конечное или счетное число непересекающихся множеств $X_1, X_2, ...$ состоящих из сообщающихся состояний и характеризующихся тем, что переходы между различными множествами невозможны.
Тогда такие множества называют \textbf{классами} или \textbf{неразложимыми классами} существенных сообщающихся состояний. 

\par\medskip
Рассмотрим Марковскую цепь с дискретным временем. Введем в рассмотрение вероятности:

$$f_{ii}^(k) = P\{\xi_k = i; \xi_l \neq i, 1 \leq l \leq k - 1 | \xi_0 = i\}$$ - вероятность первого возвращения цепи Маркова из состояния i в состояние i на k-м шаге (в дискретный момент времени k). 

$$f_{ij}^(k) = P\{\xi_k = j; \xi_l \neq j, 1 \leq l \leq k - 1 | \xi_0 = i\}$$ - вероятность первого попадания цепи в состояние j к моменту времени k из исходного состояния i.

\par\medskip

\textbf{Эргодической цепью Маркова} называется такая цепь, для которой справедливо:

$$\lim_{n \to \infty} p_{ij}^{(n)}  = \pi_{j}, j \in X, \sum_{j \in X} \pi_j = 1$$

где $\{\pi_j \}$ -эргодическое распределение.

\subsection{О средних временах переходов между состояниями} 

Обозначим через $E$ множество состояний эргодического класса, $i,j \in E$. Через $m_{ij}$ обозначим среднее число шагов, необходимых для перехода из i-го состояния в j-е состояние. Найдем это число (что для цепи с дискретным временем равносильно времени перехода из состояния i в j).
За 1 шаг цепь может из состояния i с вероятностью $p_{ij}$ перейти в состояние j, тогда время перехода будет равняться 1. Соответственно, оно будет равняться $m_{ij} = 1 + m_{kj}$, если цепь перейдет в некоторое промежуточное состояние k.
Тогда по формуле для условного математического ожидания получаем:

$$ m_{ij} = 1 \cdot p_{ij} + \sum_{k \neq j} p_{ik}(m_{kj + 1})$$

или

$$ m_{ij} = 1 + \sum_{k} p_{ik}(m_{kj + 1}) - p_{ij}m_{jj} \qquad \qquad (1)$$


Введем матричные обозначения:

$$ M = \left| m_{ij} \right|, i,j, \in E$$

\begin{equation*}
	S = \left(
	\begin{array}{cccc}
	1 & 1 & \ldots & 1\\
	1 & 1 & \ldots & 1\\
	\vdots & \vdots & \ddots & \vdots\\
	1 & 1 & \ldots & 1
	\end{array}
	\right)
\end{equation*}

\begin{equation*}
	D = \left(
	\begin{array}{cccc}
	m_{11} & 0 & \ldots & 0\\
	0 & m_{22} & \ldots & 0\\
	\vdots & \vdots & \ddots & \vdots\\
	0 & 0 & \ldots & m_{rr}
	\end{array}
	\right)
\end{equation*}

где r - число состояний данного класса. Тогда систему уравнений (1) можно переписать в матричном виде:

$$M = S + PM - PD$$

\begin{center}
	$\Downarrow$
\end{center}

$$M = (I - P)^{-1} \cdot (SPD)$$

Данная формула определяет искомые средние значения времени перехода из одного состояния в другое.

\par\medskip

Рассмотрим теперь диагональные элементы матрицы $D$. Для этого умножим систему уравнений (1) на финальную вероятность i-го состояния $\pi_i$ и просуммируем по i:

$$\sum{i} \pi_i m_{ij} = 1 + \sum_{k}m_{kj}\sum_{i}\pi_i p_{ik} - m_{jj}\sum_{i} \pi_i p_{ij}$$

Тогда, с учетом эргодической теоремы для цепей Маркова ($\pi_j = \sum_{k = 1}^{N} \pi_k p_{kj}, j = \overline {\rm 1,N}$) приходим к равенству:

$$\sum{i} \pi_i m_{ij} = 1 + \sum_{k}\pi_k m_{kj} - \pi_j m_{jj}$$

откуда находим 

$$m_{jj} = \frac{1}{\pi_j}$$

Таким образом, \textit{среднее время возвращения цепи Маркова в существенное состояние обрано пропорционально финальной вероятности этого состояния}, что, в свою очередь, означает,
что среднее время возвращения в положительное возвратное состояние конечно, а в нулевое возвратное состояние - бесконечно. 
