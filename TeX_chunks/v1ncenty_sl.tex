
					
\section{Статистическая оценка спектральной плотности гауссовской стационарной последовательности}

\textbf{Автор:} Курневич Станислав Витальевич, Б-01-008

\subsection*{Необходимые понятия}

\begin{definition}[Ширяев А.Н. <<Вероятность - 2>>~\cite{ShiryaevVeroyatnost2}, стр. 580]
	Последовательность комплексных случайных величин $\xi = (\xi_n)_{n \in \mathbb{Z}}$, с $\mathbb{E}|\xi_n^2| < \infty$, $n \in \mathbb{Z}$, являющихся единым случайным процессом, называется стационарной (в широком смысле), если для всех $n \in \mathbb{Z}$:
	\begin{enumerate}
		\item $\mathbb{E}  \xi_n = \mathbb{E} \xi_0,$
		\item $cov(\xi_{k+n}, \xi_k) = cov(\xi_n, \xi_0),~ k \in \mathbb{Z}.$
	\end{enumerate}
\end{definition}

Для простоты, не умаляя общности, в дальнейшем $\mathbb{E} \xi_0$ положим равным нулю.

\begin{definition}[Ширяев А.Н. <<Вероятность - 2>>~\cite{ShiryaevVeroyatnost2}, стр. 580]
	Функцию $R(n) = cov(\xi_n, \xi_0),~ n \in \mathbb{Z},~$ будем называть ковариационной функцией стационарной (в широком смысле) последовательности $\xi$.
\end{definition}

\begin{remark}[Ширяев А.Н. <<Вероятность - 2>>~\cite{ShiryaevVeroyatnost2}, стр. 580]
	Из определения $R(n)$ следует, что $R(n)$ является неотрицательно определенной, т.е. $\forall a_1, \dots, a_n \in \mathbb{C}$ и $\forall t_1, \dots, t_m \in \mathbb{Z},~ m \geq 1$
	\[\sum\limits_{i,j = 1}^{m}{a_i \overline{a}_j R(t_i - t_j)} \geq 0.\]
\end{remark}

\begin{theorem}[Герглотц)(Ширяев А.Н. <<Вероятность - 2>>~\cite{ShiryaevVeroyatnost2}, стр. 585]
Пусть $R(n)$ --- ковариационная функция стационарной (в широком смысле) случайной последовательности с нулевым средним. Тогда на $([-\pi,\pi),~ \mathscr{B}([-\pi,\pi)))$ найдется такая конечная мера $F = F(B), B \in \mathscr{B}([-\pi;\pi))$, что для любого $n \in \mathbb{Z}$
\begin{equation}
	R(n) = \int\limits_{-\pi}^{\pi}e^{i\lambda n}F(d\lambda),
\end{equation}
где интеграл $\int\limits_{-\pi}^{\pi}{e^{i\lambda n}F(d\lambda)}$ понимается как интеграл Лебега-Стильеса по множеству $[-\pi,\pi).$
\end{theorem}

\subsection*{Постановка задачи}

Пусть $\xi = (\xi_n),~ n \in \mathbb{Z},$ стационарная в широком смысле (действительная --- для простоты) случайная последовательность с математическим ожиданием $\mathbb{E} \xi_n = m$ и ковариацией $R(n) = \int\limits_{-\pi}^{\pi}{e^{i \lambda n} F(d \lambda)}$.

Пусть $x_0,~ x_1,~ \ldots,~ x_{N-1}$ --- полученные в ходе наблюдений значения случайных величин $\xi_0,~ \xi_1,~ \ldots,~ \xi_{N-1}.$
В качестве оценки $R(n)$ по результатам $N$ наблюдений будем брать величину
\[\hat{R}_N(n;x) = \frac{1}{N - n} \sum\limits_{k=0}^{N - n - 1}{x_{n+k}x_{k}},~ 0 \leq n < N.\]
При оценке спектральной плотности $f(\lambda)$ будем предполагать $m = 0$.

\subsection*{Оценка спектральной плотности}
Перейдем теперь к вопросу построения оценки для спектральной плотности $f(\lambda)$ (в предположении, что она существует).
Положим для $N \geq 1$ и $\lambda \in [-\pi;\pi]$
\begin{equation}
	f_N(\lambda) = \frac{1}{2\pi N} \sum\limits_{k=1}^{N} \sum\limits_{l=1}^{N}{R(k - l)e^{-ik\lambda} e^{il\lambda}.}
	\label{f_lambda_raw_representation}
\end{equation}
В силу неотрицательной определенности $R(n)$ функция $f_N(\lambda)$ неотрицательна. Поскольку число тех пар $(k,~l)$, для которых $k-l=m$, есть $N - |m|$, то
\begin{equation}
	f_N(\lambda) = \frac{1}{2\pi} \sum\limits_{|m| < N}{\left(1 - \frac{|m|}{N}\right)R(m) e^{-im\lambda}.}
	\label{f_lambda_representation}
\end{equation}
Из доказательства теоремы Герглотца следует, что построенная по $f_N(\lambda)$ функция 
\[F_N(\lambda) = \int\limits_{-\pi}^{\lambda}{f_N(\nu)d\nu}\]
сходится в основном к спектральной функции $F(\lambda)$. 
Поэтому, если $F(\lambda)$ имеет плотность $f(\lambda)$, то для каждого $\lambda \in [-\pi, \pi)$ выполнено:
\begin{equation}
	\int\limits_{-\pi}^{\lambda}{f_N(\nu)d\nu} \to \int\limits_{-\pi}^{\lambda}{f(\nu)d\nu.}
	\label{f_n_to_f}
\end{equation}
Исходя из представления $f_N(\lambda)$ в виде $(\ref{f_lambda_representation})$, факта $(\ref{f_n_to_f})$ и, вспомнив, что в качестве оценки $R(n)$ (по наблюдениям $x_0,~ x_1, \dots,~ x_{N-1}$) брались величины $\hat{R}_N(n;x)$, возьмем в качеcтве оценки $f(\lambda)$ функцию

\begin{equation}
	\hat{f}_N(\lambda, x) = \frac{1}{2\pi} \sum\limits_{|n| < N}{\left(1 - \frac{|n|}{N}\right)\hat{R}_N(n; x)e^{-i\lambda n}},
\end{equation}
полагая $\hat{R}_N(n;x) = \hat{R}_N(|n|;x).$

Функцию $\hat{f}(\lambda; x)$ принято называть \textit{периодограммой}, ее также можно представить в виде:
\begin{equation}
	\hat{f}_N(\lambda, x) = \frac{1}{2\pi N}{\left|\sum\limits_{k=0}^{N-1}{x_n e^{-i \lambda n}}\right|^2.}
\end{equation}
Поскольку $\mathbb{E} \hat{R}_N(n; \xi) = R(n),~ |n| < N$, то
\begin{equation}
	\mathbb{E} \hat{f}_N(\lambda;\xi) = f_N(\lambda).
	\label{math_expectation_f_N}
\end{equation}

Если спектральная функция $F(\lambda)$ имеет плотность $f(\lambda)$, то, учитывая, что $f_N(\lambda)$ может быть записана в виде $(\ref{f_lambda_raw_representation})$, находим, что
\[f_N(\lambda) = \frac{1}{2\pi N} \sum\limits_{k=0}^{N-1} \sum\limits_{l = 0}^{N-1}{\int\limits_{-\pi}^{\pi}{e^{i\nu(k-l)} e^{i\lambda(l-k)}f(\nu) d\nu}} = \int\limits_{-\pi}^{\pi}{\frac{1}{2\pi N} \left|\sum\limits_{k=0}^{N-1}{e^{i(\nu - \lambda)k}}\right|^2 f(\nu) d\nu}.\]
Функция
\[\Phi_N(\lambda) = \frac{1}{2\pi N} \left|\sum\limits_{k=0}^{N-1}{e^{i\lambda k}}\right|^2 = \frac{1}{2\pi N} \left|\frac{\sin{\frac{\lambda}{2}N}}{\sin{\frac{\lambda}{2}}}\right|^2\]
называется ядром Фейера. Из свойств ядра Фейера известно, что почти для всех $\lambda$ (по мере Лебега) верно
\begin{equation}
	f_N(\lambda) \equiv \int\limits_{-\pi}^{\pi}{\Phi_N(\lambda - \nu)f(\nu)d\nu} \to f(\lambda).
	\label{fejer_kernel_limit}
\end{equation}
Из $(\ref{math_expectation_f_N})$ и $(\ref{fejer_kernel_limit})$ следует, что для почти всех $\lambda \in [-\pi,\pi)$
\begin{equation}
	\mathbb{E} \hat{f}_N(\lambda; \xi) \to f(\lambda),
	\label{feyer_consequence}
\end{equation}
иначе говоря, оценка $\hat{f}_N(\lambda; x)$ спектральной плотности $f(\lambda)$ по наблюдениям $x_0,~ x_1,~ \dots,~ x_{N-1}$ является \textit{асимптотически несмещенной}.

Однако на индивидуальных наблюдениях $x_0,~ x_1,~ \dots,~ x_{N-1}$ значения $\hat{f}_N(\lambda;x)$ оказываются далекими от истинных значений $f(\lambda)$.

Для примера можно рассмотреть $\xi = (\xi_n)$ --- стационарную последовательность независимых гауссовских случайных величин, $\xi_n \thicksim \mathcal{N} (0,1)$. Тогда $f(\lambda) \equiv \frac{1}{2\pi}$, а
\[\hat{f}_N(\lambda; \xi) = \frac{1}{2\pi} \left|\frac{1}{\sqrt{N}}\sum\limits_{k=0}^{N-1}{\xi_k e^{-i \lambda k}}\right|^2.\]
Поэтому $2\pi \hat{f}_N(0; \xi)$ по распределению совпадает с квадратом гауссовской случайной величины $\eta \thicksim \mathcal{N} (0,1)$. Отсюда при любом $N$
\[\mathbb{E} |\hat{f}_N(0; \xi) - f(0)|^2 = \frac{1}{4\pi^2} \mathbb{E} |\eta^2 - 1|^2 > 0.\]

Отсюда понятно, что периодограмма не может служить удовлетворительной оценкой спектральной плотности. Поэтому в качестве оценки для $f(\lambda)$ часто используют оценки вида
\begin{equation}
	f_N^{\mathds{W}}(\lambda; x) = \int\limits_{-\pi}^{\pi}{\mathds{W}_N(\lambda - \nu)\hat{f}_N(\nu; x) d\nu},
	\label{estimation_w}
\end{equation}
которые строятся по периодограмме $\hat{f}_N(\lambda; x)$ и некоторым <<сглаживающим>> функциям $\mathds{W}_N(\lambda)$, называемым \textit{спектральными окнами}.
Требования на функции $\mathds{W}_N(\lambda)$ состоят в том, чтобы:
\begin{enumerate}
	\item $\mathds{W}_N(\lambda)$ имели резко выраженный максимум в окрестности точки $\lambda = 0;$
	\item $\int\limits_{-\pi}^{\pi}{\mathds{W}_N(\lambda)d\lambda} = 1;$
	\item $\mathbb{E} |\hat{f}_N^\mathds{W}(\lambda; \xi) - f(\lambda)|^2 \to 0,~ N \to \infty,~ \lambda \in [-\pi;\pi).$
\end{enumerate}
В силу ($\ref{feyer_consequence}$) и требования 2 оценки $\hat{f}_N(\lambda; x)$ являются асимптотически несмещенными. Требование 3 является условием асимптотической состоятельности в среднеквадратичном смысле, что, как было показано выше, нарушается для периодограммы. Требование 1 обеспечивает <<вырезание>> из периодограммы требуемой частоты $\lambda$.

Приведем некоторые примеры оценок вида ($\ref{estimation_w}$).

\textit{Оценка Бартлета} основана на выборе спектрального окна
\[\mathds{W}_N(\lambda) = a_N B(a_N \lambda),\]
где $a_N \uparrow \infty,~ \frac{a_N}{N} \to 0,~ N \to \infty$ и
\[B(\lambda) = \frac{1}{2\pi}\left|\frac{\sin{\frac{\lambda}{2}}}{\frac{\lambda}{2}}\right|^2.\]

\textit{Оценка Парзена} использует в качестве спектрального окна функцию
\[\mathds{W}_N(\lambda) = a_N P(a_N \lambda),\]
где $a_N$ такие же, что и выше, а
\[P(\lambda) = \frac{3}{8\pi}\left|\frac{\sin{\frac{\lambda}{4}}}{\frac{\lambda}{4}}\right|^4.\]

