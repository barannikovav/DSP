
\section{Метод продолжения меры с кольца на сигма-кольцо с помощью пополнения полуметрических пространств}

\textbf{Автор:} Глухарева Дарья Вадимовна, Б-01-004

\subsection{Понятие полуметрического пространства} 
(Богачев " Основы теории меры") \\
Полуметрическое пространство - это множество X вместе с функцией

$d : X\times X \rightarrow [0, \infty),$ 

которая удовлетворяет следующим свойствам: для любых $x, y, z$ накладываются следующие условия:

$-x=y$, тогда $d(x, y) = 0$; 

$-d(x, y)$ = $d(y, x)$; 

$-d(x, z) \leq d(x, y) + d(y, z)$.
\subsection{Пополнение полуметрического пространства}
Пополнение полуметрического пространства (X, d) - это полное полуметрическое пространство (Y, D), такое что X является плотным подмножеством Y относительно полуметрики d, а функция D|X × X = d.

\subsection{Продолжение меры с кольца на сигма-кольцо}

(Богачев "Основы теории меры") \\
\begin{definition} Пара (X,A),состоящая из множества X и $\sigma-алгебры$ A его подмножеств, называется измеримым пространсвом.
	\end{definition}
 
\begin{definition} Множество E называется измеримым относительно внешней меры $\mu^*$, порожденной счетно-аддитивной мерой $\mu$ на алгебре A, если

$$\mu^*(E) + \mu^*(X\setminus E) = \mu(X).$$
\end{definition}

Класс измеримых множеств обозначим через $A_\mu$.

\begin{lemma} Пусть $\mu^*$ - неотрицательная аддитивная функция множества на алгебре A. Имеем (i) $\mu^* (\emptyset) = 0$,

$$(ii)\  \mu^*(A) \leq \mu^*(B),\  A \subset B, $$

$$(iii)\  \mu^*(\bigcup_{n= 1}^{+\infty} E_n) \leq \sum_{n= 1}^{+\infty} \mu^* (E_n)\ $$ для всех $E_n \subset X$. 
\end{lemma}

\begin{proof} Первые два свойства очевидны. Для доказательства (iii) возьмем $\epsilon > 0$ и для каждого n покроем $E_n$ множествами $A_{nj} \in A$ так, что

$$ \sum_{j=1}^{+\infty} \mu(A_{nj} ) \leq \mu^* (E_n ) + \epsilon2^{-n} .  $$
Это можно, если помнить, что такое inf. Тогда $E =\bigcup_n$ $E_n$  покрыто всеми $A_{nj}$ и

$$ \mu^*(E) \leq  \sum_{n= 1}^{+\infty} \mu^*(A_{n,j} ) \leq \sum_{n= 1}^{+\infty}(\mu^*(E_n) + \epsilon2^{-n}) \leq \sum_{n= 1}^{+\infty} \mu^*(E_n) + \epsilon, $$
что дает нужную оценку при $\epsilon \rightarrow 0.$
\end{proof}


\begin{lemma} Имеют место соотношения 

 $\mu^*$(A)= $\lim\limits_{n\to \infty}$ $\mu(A_n)$ , если $A_n \in A$ , $A_n$ $\subset$ $A_{n+1}$, $A=\bigcup_n$ $A_n$,

$$ \mu^*(A\cup B)+\mu^*(A \cap B)\leq \mu^*(A)+\mu^*(B)$$ для всех A,B.
\end{lemma}

\begin{proof}
Ясно, что $\mu(A_n) \leq \mu^*$(A). Пусть $\epsilon$ > 0.Найдем такие возрастающие номера $n_k$, что $\mu$($A_{nk}$ ) > $\lim\limits_{n\to \infty} \mu(A_n) - \epsilon 2^{-k}$. 
Тогда $\mu(A_{nk+1}\setminus A_{nk})$ < $\epsilon 2^{-k}$, причем множества $A_{n1}, A_{n2}\setminus A_{n1}$ и т.д. покрывают A. Поэтому
$$\mu^*(A)\leq\mu(A_{n1})+\mu(A_{n2}\setminus A_{n1})+···\leq \mu(A_{n1})+\epsilon \leq \lim\limits_{n\to \infty} \mu(A_n)+\epsilon, $$
что ввиду произвольности $\epsilon$ дает нужное равенство.

Для доказательства оставшейся оценки возьмем $\epsilon$ > 0 и найдем
множества U и V вида $ U = \bigcup_n U_n, V = \bigcup_n V_n, где U_n,V_n \in A, U_n \subset U_{n+1}, V_n \subset V_{n+1},$ для которых $\mu^*(A) \geq \mu^*(U) - \epsilon, \mu^*(B) \geq \mu^*(V ) - \epsilon.$ Это можно сделать по определению внешней меры: покрыв U множествами $I_j \in A $ c $ \mu^*(U) \geq \sum \mu(I_j) - \epsilon$ и взяв $U_n = \bigcup_n I_j$ и аналогично для V . Тогда $A \cup B \subset U \cup V , A \cap B \subset U \cap V $, значит,

$$\mu^*(A \cup B) + \mu^*(A \cap B) \leq \mu^*(U \cup V ) + \mu^*(U \cap V ).$$
По доказанному,
$$\mu^*(U\cup V)= \lim\limits_{n\to \infty} \mu(U_n\cup V_n), \mu^*(U\cap V)= \lim\limits_{n\to \infty} \mu(U_n\cap V_n).$$
В силу аддитивности меры на A имеем
$$\mu(U_n \cup V_n) + \mu(U_n \cap V_n) \leq \mu(U_n) + \mu(V_n).$$
Так как $\mu(U_n) \rightarrow \mu^*(U), \mu(V_n) \rightarrow \mu^*(V ),$ получаем
$$\mu^*(U \cup V ) + \mu^*(U \cap V ) \leq \mu^*(U) + \mu^*(V ) \leq \mu^*(A) + \epsilon + \mu^*(B) + \epsilon.$$
Вспоминая, что левая часть не меньше $\mu^*(A\cup B)+\mu^*(A\cap B), при \epsilon \rightarrow 0$ получаем нужную оценку. 
\end{proof}

\begin{theorem} Класс $A_\mu$ является $\sigma$-алгеброй, содержит A, поэтому и $\sigma$(A), внешняя мера $\mu^*$ счетно-аддитивна на $A_\mu$ и продолжает $\mu$.
\end{theorem}

\begin{proof} Уже знаем, что $A \subset A_\mu$ и $\mu^*$ продолжает $\mu$. Если бы знать, что $A_\mu$ — алгебра, а $\mu^*$ аддитивна, то все быстро проверяется: если $A_n \in A_\mu$ дизъюнктны, то для $A =\cup A_n $ при всех N имеем

$$\sum_{n= 1}^{N} \mu^*(A_n) = \mu^*(\bigcup_{n= 1}^{N} A_n) \leq \mu^*(A) \leq \sum_{n}\mu^*(A_n),$$

откуда $\mu^*(A) = \sum_{n} \mu^*(A_n).$ Кроме того, $X\setminus A \subset X \setminus$ $\bigcup_{n= 1}^{N}$  $A_n$ для всякого N, поэтому

$$\mu^*(A)+\mu^*(X\setminus A) \leq \sum_{n= 1} \mu^*(A_n)+\mu(X)- \sum_{n= 1}\mu^*(A_n) = \mu(X)+ \sum_{n= 1} \mu^*(A_n),$$

что при $N \rightarrow \infty$ дает оценку $\mu^*(A) + \mu^*(X\setminus A) \leq \mu(X).$ По лемме есть и противоположная оценка. Это означает измеримость A, т.е. замкнутость относительно и счетных объединений.

Проверим, что $A_\mu$ есть алгебра. Дополнения можно брать по определению. Пусть $A, B \in A_\mu$. Надо показать, что

$$\mu^*(A \cup B) + \mu^*(X\setminus(A \cup B)) \leq \mu(X),$$
так как обратное верно по лемме. Мы имеем равенство

$$\mu^*(A) + \mu^*(X\setminus A) + \mu^*(B) + \mu^*(X\setminus B) = 2\mu(X),$$
а также неравенства из последней леммы
$$\mu^*(A \cup B) + \mu^*(A \cap B) \leq \mu^*(A) + \mu^*(B),$$
$$\mu^*(X\setminus(A \cup B)) + \mu^*(X\setminus(A \cap B)) \leq \mu^*(X\setminus A) + \mu^*(X\setminus B),
$$
где во второй оценке использованы тождества
$$X\setminus(A \cup B) = (X\setminus A) \cap (X\setminus B), X\setminus(A \cap B) = (X\setminus A) \cup (X\setminus B).$$

Сумма левых частей последних двух неравенств равна $2\mu(X),$ что возможно только в том случае, когда эти неравенства обращаются в равенства, ибо $\mu^*(A\cup B) + \mu^*(X\setminus(A \cup B)) \geq \mu(X)$ и аналогично для пересечения. Итак, $A_\mu$ алгебра. Остается проверить, что $\mu^*$$(A \cup B) =$ $\mu^*$(A) + $\mu^*$(B) для дизъюнктных A, B $\in$ $A_\mu$. Это видно из второго из предыдущих равенств (в которые превратились неравенства), ибо для измеримых A, B оно принимает вид $\mu^*(X\setminus(A \cup B)) + \mu(X) = 2\mu(X) - \mu^*(A) - \mu^*(B)$, где слева с учетом измеримости $A \cup B$ имеем 2$\mu(X)$ - $\mu^*(A \cup B).$
\end{proof}

\begin{theorem}[Лекция 5] Ограниченная неотрицательная счетно-аддитивная мера $\mu : K \rightarrow [0, N]_R$ на кольце множеств (K) имеет единственное счетно-аддитивное продолжение на \ae$_{\sigma}(K)$\\
	\end{theorem}
\begin{proof}
Пусть $\rho: K\times K \rightarrow [0,N]_R \subset [0,\infty)_R$ задается формулой:\\ $\forall (A,B) \in K\times K \  \rho(A,B):=\mu(A\triangle B) = \mu(B\triangle A) = \rho (B,A)$\\
неравенство треугольника: если еще $C \in K$, то $A\triangle B \subset (A \triangle C) \cup (C\triangle B)$\\
$x\in A\triangle B: 1) x\in A\backslash B,\  2) x\in B\backslash A$\\
$((A\backslash C)\cup (C\backslash A)) \cup ((C\backslash B)\cup (B\backslash C))$\\
$\rho(A,B) = \mu(A\triangle B) \leq \mu((A\triangle C)\cup (C\triangle B))\leq \mu(A\triangle C) + \mu(C\triangle B) = \rho(A,C) + \rho(C,B)$\\
\end{proof}

\begin{theorem} Любое полуметрическое пространство имеет пополнение. \\
	\end{theorem}
\begin{proof} Пусть $\{M ; \rho\}$ — заданное полуметрическое пространство. Построим новое полуметрическое пространство $\{M^*;\rho^* \}$, которое содержит пространство $\{M;\rho \}$.
Последовательности $\{x_n\}, \{y_n\}$ элементов из M, удовлетворяющих условию
$$\lim\limits_{n\to \infty} \rho(x_n,y_n) = 0$$
будем называть эквивалентными и писать $\{x_n\} \sim \{y_n\}$. Очевидно, если $\{x_n\} \sim \{y_n\}, а \{y_n\} \sim \{z_n\}$, то $\{x_n\} \sim \{z_n\}$, и поэтому множество всех последовательностей в M распадается на непересекающиеся классы эквивалентных последовательностей. Нас будут интересовать только фундаментальные последовательности.\\
Итак, в пространстве $\{M ; \rho\}$ рассмотрим множество всех фундаментальных последовательностей, и через $M^*$ обозначим множество всех классов эквивалентных фундаментальных последовательностей. Если фундаментальная последовательность $\{x_n\}$ принадлежит классу $x^*$, то, как обычно, будем писать $\{x_n\} \in x^*$. В множестве $M^*$ определим метрику. Очевидно,
$$\rho(x_n,y_n) \leq \rho(x_n,x_m) + \rho(x_m,y_m) + \rho(y_m,y_n)$$ \\
для любых n и m, и поэтому
$$|\rho(x_n,y_n) - \rho(x_m,y_m)| \leq \rho(x_n,x_m) + \rho(y_m,y_n).$$\\
Отсюда следует, что если последовательности $\{x_n\}$ и $\{y_n\}$ фундаментальные, то числовая последовательность $\rho(x_n,y_n)$, $n \in \mathds{N}$,- тоже фундаментальная и, следовательно, имеет предел. Тогда если $\{x_n\} \in x^*,$ $\{y_n\} \in y^*$, то расстояние между $x^* $ и $ y^*$ определим по формуле
$\rho(x^*,y^*) = \lim \rho(x_n,y_n).$\\
Прежде всего покажем, что это определение не зависит от выбора последовательностей из классов $x^*$ и $y^*.$\\
Пусть $\{x'_n\} \in x^* $ и $ \{y'_n\} \in y^*$. Тогда для любого $ n \in \mathds{N}$

$$ \rho(x'_n,y'_n ) \leq \rho(x'_n,x_n) + \rho(x_n,y_n) + \rho(y_n,y'_n ),$$
$$\rho(x_n,y_n) \leq \rho(x_n,x'_n) + \rho(x'_n,y'_n ) + \rho(y_n' ,y_n),$$
и поэтому
$$ |\rho(x'_n,y'_n ) - \rho(x_n,y_n)| \leq \rho(x_n,x'_n) + \rho(y_n,y'_n ).$$\\
Отсюда следует, что
$$\lim\limits_{n\to \infty} \rho(x'_n,y'_n)= \lim\limits_{n\to \infty} \rho(x_n,y_n).$$\\
Функция $\rho(x^*,y^*)$ удовлетворяет всем аксиомам метрики. Действительно, $\rho(x^*,y^*) \geq 0$ и $\rho(x^*,y^*) = \rho(y^*,x^*)$ для любых $x^*$ и $y^*$ из $M^*$. Далее, если $\rho(x^*,y^*) = 0,$ то $x^*$ и $y^*$ совпадают, так как в этом случае, если $\{x_n\} \in x^*, \{y_n\} \in y^*,$ то $\{x_n\} \sim \{y_n\}.$ Наконец, если $\{x_n\} \in x^*, \{y_n\} \in y^*, \{z_n\} \in z^*,$ то из неравенства
$$\rho(x_n,y_n) \leq \rho(x_n,z_n) + \rho(z_n,y_n)$$\\
в пределе при $ n \rightarrow \infty$ получаем\\
$$\rho(x^*,y^*) \leq \rho(x^*,z^*) + \rho(z^*,y^*).$$\\
Итак, построено полуметрическое пространство $\{M^*;\rho^* \}$, элементами которого являются классы эквивалентных фундаментальных последовательностей элементов из M.\\
Покажем, что пространство $\{M^*;\rho^* \}$ содержит подпространство, которое изометрично пространству $\{M;\rho \}$.\\
Каждому элементу $x \in M$ поставим в соответствие элемент $x^* \in M^*$, содержащий стационарную последовательность $x_n = x$, $n \in \mathds{N}.$ Очевидно, это соответствие определяет взаимно однозначное отображение M на некоторое подмножество M' множества $M^*.$ Более того, это отображение является изометричным, так как если $x^*$ и $y^*$ из M', то существуют x и y из M такие, что $\{x\} \in x^*$, $\{y\} \in y^*$, и поэтому\\
$$\rho^*(x^*,y^*) = \rho(x,y).$$\\
Докажем, что множество M' плотно в $M^*$ , т.е. что любая точка $x^* \in M^*$ является пределом последовательности из M'.\\
Пусть $\{x_n\} \in x^*$. Через $x^*_k$ обозначим элемент из M', соответствующий элементу $x_k \in M.$ Тогда, согласно определению,\\
$$\rho^*(x^*,x^*_k) = \lim\limits_{n\to \infty} \rho(x_n,x_k).$$\\
А так как последовательность $\{x_n\}$ фундаментальная, то\\
$$\forall \epsilon \textgreater 0 \   \exists N_\epsilon: \forall n,k\geq N_\epsilon \   \rho(x_n,x_k)\textless \epsilon/2,$$\\
и поэтому
$$\forall k \geq N_\epsilon  \  \rho^*(x^*,x^*_k) \leq \epsilon /2 \textless \epsilon.$$\\
Следовательно, $\lim\limits_{k\to \infty} \rho^*(x^*,x^*_k) = 0.$
Для завершения доказательства осталось показать, что полуметрическое пространство $\{M^*;\rho^* \}$ полное.\\
Пусть $\{x^*_n\}$ — фундаментальная последовательность точек из $M^*.$ Для любого $n \in \mathds{N}$ существует $y_n \in M$ такое, что\\
$$\rho^*( x^*_n ; y^*_n ) \textless 1/n ,$$\\
где $y^*_n$ — элемент из M', соответствующий элементу $y_n \in M.$\\
Последовательность $\{y^*_n\}$ фундаментальная. Действительно, это следует из неравенства\\
$$ \rho^*(y^*_n,y^*_m ) \leq \rho^*(y^*_n,x^*_n ) + \rho^*(x^*_n,x^*_m ) + \rho^*(x^*_m,y^*_m ) \textless 1/n + \rho^*(x^*_n,x^*_m ) + 1/m$$
и фундаментальности последовательности $\{x_n\}$. А так как $\rho^*(y^*_n,y^*_m ) = \rho(y_n,y_m),$ то фундаментальной будет и последовательность $\{y_n\}$. Через $y^*$ обозначим класс эквивалентных фундаментальных последовательностей, содержащий последо- вательность $\{y_n\}$. Тогда
$$\rho^*(y^*,x^*_n ) \leq \rho^*(y^*,y^*_n ) + \rho^*(y^*_n,x^*_n ) \textless \rho^*(y^*,y^*_n ) + 1/n = \lim\limits_{k\to \infty} \rho^*(y_k,y_n) + 1/n .$$
А так как\\
$$\forall \epsilon \textgreater 0 \   \exists N_\epsilon: \forall n,k\geq N_\epsilon \   \rho(x_k,y_n) + 1/n \textless \epsilon/2,$$\\
то
$$\forall n\geq N_\epsilon \   \rho^*(y^*,x^*_n) \leq \epsilon/2 \textless \epsilon$$\\
и, следовательно, $\lim\limits_{n\to \infty} \rho^*(y^*,x^*_n) = 0.$
\end{proof}

\begin{remark}[Лекция 8] $\rho_{\bar{\mu}}$ порождает ту же самую метрику на фактор пространстве $\bar{K}/\tilde{\rho}_{\bar{\mu}}$, что и копия аналогично получаемой метрики из пополнения полуметрического пространства $(K, \rho_{\mu})$, т.е. классу эквивалентных фундаментальных последовательностей будет соответствовать один класс эквивалентных измеримых множеств из $(\bar{K}^{\mu}, \rho_{\bar{\mu}})$.
	\end{remark}
