

\section{Стационарный белый шум с непрерывным временем}

\textbf{Автор:} Роман Соколов, Б-01-005

\subsection{Обобщённые стационарные случайные процессы}

\paragraph{Напоминание.}
Обобщённая функция -- непрерывный линейный функционал, определённый на некотором пространстве основных функций. 
Рассмотрим пространство основных функций S. Оно состоит из бесконечно дифференцируемых функций $\phi(t)$, убывающих на бесконечности быстрее любой степени вместе с их производными:
$$\forall m \geq 0 \mapsto (1 + |t|^m)\phi^{(k)}(t) \longrightarrow 0, k = 0, 1, ...$$
\paragraph{Определение. } Обобщёнными случайным процессом $\xi = \xi(\phi)$ будем называть непрерывное линейное отображение $\xi:\phi \longrightarrow L_{\Omega}^2$ множества основных функций $\phi$ в множество случайных величин $L_{\Omega}^2$, где $L^2_{\Omega}$ -- гильбертово пространство случайных величин, суммируемых с квадратом.

Если $\xi(t)$ -- обычный случайный процесс с кореляционной функцией B(s, t), (при $T = (-\infty, +\infty)$ возникает и отображение  $\xi \longrightarrow \xi(\phi)$, т.е. обобщённый случайный процесс.

Получить обобщённые случайные процессы, не сводящиеся к обыкновенным, можно, например, операцией дифференцирования. Обобщённый случайный процесс можно дифференцировать сколь угодно раз, пользуясь определением $$(\xi', \phi) = -(\xi, \phi').$$ После того как обыкновенный процесс продиффернцировали столько раз, сколько позволяет гладкость корреляционной функции, дальнейшие производные будут обощёнными процессами.

\paragraph{Определения. } 
\begin{enumerate}
    \item Математическое ожидание
$$m(\phi) = \textbf{M}\xi(\phi)$$
(это линейный, в силу линейности $\xi (\phi)$ функционал)
    \item Корреляционный функционал
$$B(\phi, \psi) = \textbf{M}\xi(\phi)\overline{\xi(\psi)}$$
(это билинейный эрмитов функционал: $B(\phi, \psi) = \overline{B(\phi, \psi)}$
    \item Обобщённый стационарный процесс -- это отображение  $\xi:\phi \longrightarrow L_{\Omega}^2$ пространства $S = {\phi}$ основных функций в $L_{\Omega}^2$, для которого
    $$\textbf{M}\xi(\phi) = m \int_{-\infty} ^ {+\infty}\phi(t)dt, m - \text{число}$$
    $$\textbf{M}|\xi(\phi)|^2 = (B, \phi * \overline{\phi(-t)})$$
\end{enumerate}

\subsection{Винеровский процесс}
Как известно, броуновским движением называется наблюдаемое под микроскопом движение мелких частиц, взвешенных в жидкости. Построим его математическую модель. Пусть $\omega(t)$ -- абсцисса движущейся точки в момен $t, t \geq 0$. Для простоты $\omega(0) = 0$. Броуновская частица ведёт себя столь нерегулярно, что для любых значений $t_1 < t_2 < ... < t_n$ приращения процесса $\omega(t)$, т.е. разности
$$\omega(t_1) - \omega(t_0) = \omega(t_1), ..., \omega(t_n) - \omega(t_{n - 1}),$$ естественно считать независимыми случайными величинами. Предположим дополнительно, что распределение любой из разностей $\omega(t_{i + 1}) - \omega(t_{i})$ нормально с нулевым математическим ожиданием и дисперсией $\sigma^2(t_{i + 1} - t_{i})$, где $\sigma$ - некоторый параметр. Этим предположением задаётся совместное рапределение разностей: как распределение независимых случайных величин. Следовательно, задаётся и распределение величин
$$\omega(t_1), ..., \omega(t_n)$$
как функций от величин разностей, т.е. конечномерное распределение процесса $\omega(t)$. Случайный процесс с такими конечномерными распределениями называется винеровским процессом. 

\subsection{Белый шум}
Построим с помощью винеровского процесса случайную меру с ортогональными значениями.

Выпустим из точки $\omega(0) = 0$ две независимые реализации винеровского процесса $\omega_1(t)$ и $\omega_2(t)$, $t \geq 0$. На оси значений $\lambda: -\infty < \lambda < \infty$ введём функцию $Z(\lambda)$ по следующему правилу: $Z(0) = 0$, для $\lambda > 0$
положим $Z(\lambda) = \omega_1(\lambda)$, а для $\lambda < 0$
положим $Z(\lambda) = -\omega_2(-\lambda) = -\omega_2(|\lambda|)$. Для отрезка $[\lambda_1, \lambda_2], \lambda_1 < \lambda_2$ положим
$$Z\{[\lambda_1, \lambda_2]\} = Z(\lambda_2) - Z(\lambda_1)$$
(для $0 \leq \lambda_1 < \lambda_2$ случайная мера $Z\{[\lambda_1, \lambda_2]\} = \omega_1(\lambda_2) - \omega_1{\lambda_1}$);
для $\lambda_1 < \lambda_1 \leq 0$ случайная мера $Z\{[\lambda_1, \lambda_2]\} = \omega_2(|\lambda_1|) - \omega_2(|\lambda_2|)$, а если $\lambda_1 < 0 < \lambda_2$, то мера $Z\{[\lambda_1, \lambda_2]\}$ равна сумме мер отрезков $[\lambda_1, 0]$ и $[0, \lambda_2]$). При этом $\textbf{M}|Z\{[\lambda_1, \lambda_2]\}|^2 = \sigma^2|\lambda_1 - \lambda_2|$, $\textbf{M}Z\{[\lambda_1, \lambda_2]\}$ = 0, а для непересекающихся отрезков $[\lambda_1, \lambda_2]$ и $[\lambda_1', \lambda_2']$ значения меры $Z$ независимы, следовательно, и ортогональны.

Можно продолжить меру Z на борелевские подмножества A, получив, что 
$$F(A) = \textbf{M}|Z(A)|^2 = \sigma^2 l(A),$$ $l(A)$ - лебегова мера (длина A). Мера F(A) есть мера степенного роста: 
$$\int_{-\infty}^{+\infty}\frac{F(d\lambda)}{1 + \lambda ^ 3} = \int_{-\infty}^{+\infty}\frac{\sigma^2(d\lambda)}{1 + \lambda ^ 3} < \infty$$

Следовательно можно рассмотреть обобщённый случайный процесс $\xi$ со спектральной мерой Z. Этот процесс называется белым шумом интенсивности $\sigma^2$. Его корреляционный функционал B является преобразованием Фурье от меры $\sigma ^2l(A)$, т.е. равен $2\pi \sigma^2 \delta(t)$. Следовательно $$\textbf{M}|\xi(\phi)| ^ 2 = (B, \phi * \overline{\phi(-t)}) = 2\pi\sigma^2 \int_{-\infty}^{+\infty} |\phi(t)|^2 dt = 2\pi\sigma^2||\phi||^2,$$ где $||\phi||^2$ понимается в смысле нормы в $L^2$ по мере Лебега.

Если на всей прямой $-\infty < t < \infty$ определить винеровский процесс $\omega(t)$ (так же, как это было сделано для функции $Z(\lambda)$) c параметром $2\pi\sigma^2$, то случайный процесс $\xi(t)$ можно понимать как обобщённую производную $\omega(t)$. Действительно, для гладкой финитной функции $\phi = \phi(t)$

\begin{multline*}
  (\omega', \phi) = -(\omega, \phi') = -\int_{-\infty}^{+\infty} \omega(t)\phi'(t)dt = -\int_{-\infty}^{+\infty}\omega(t)d\phi(t) = \\ = -lim \sum_{i}\omega(t_i)(\phi(t_{i+1}) - \phi(t_{i})) = lim \sum_{i}\phi(t_i)(\omega(t_{i}) - \phi(t_{i-1}))  
\end{multline*}
откуда 
$$\textbf{M}|(\omega', \phi)|^2 = 2\pi\sigma^2 lim \sum_{i}|\phi(t_i)|^2 (t_i - t_{i-1}) = 2\pi\sigma^2\int_{-\infty}^{+\infty} |\phi(t)|^2 dt, $$
$$\textbf{M}(\omega', \phi) = 0$$

Итак, белый шум есть обобщённая производная от винеровского процесса. Поскольку приращения винеровского процесса в различные моменты времени независимы, говорят еще, что белый шум есть процесс с независимыми значениями. его спектральная плотность равна константе.

Часто на практике случайный процесс, о котором мало что известно, считают белым шумом. Также пользуются белым шумом в полосе частот $\Lambda$: считают, что $f(\lambda) = C$ при $|\lambda| \leq \Lambda$ и 
$f(\lambda) = 0$ при $|\lambda| > \Lambda$. Нелишне заметить, что белый шум в полосе частот -- это обыкновенный процесс, реализации которого $\xi(t)$ являются аналитическими функциями t. Поэтому не следует абсолютизировать эту модель(как и модель белого шума, значения которго при каждом t бесконечны).
