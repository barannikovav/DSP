
\section{Стационарный в широком смысле случайный процесс с непрерывным и дискретным временем}

 \textbf{Автор:} Мухина Анфиса Дмитриевна , Б-01-004

\subsection{Стационарная в широком смысле последовательность}

\emph{[А.Н. Ширяев ''Вероятность'', стр. 403 - 404]}

Последовательность комплексных случайных величин $\xi = (\xi_{n})_{n \in \mathbb {Z}}$ с $M|\xi_{n}|^2 < \infty$, $n \in \mathbb {Z}$, называется стационарной в широком смысле, если для всех $n \in \mathbb {Z}$ $$M\xi_{n} = M\xi_{0},$$
$$cov(\xi_{k+n}, \ \xi_{k}) = cov(\xi_{n}, \ \xi_{0}), \ k \in \mathbb {Z}.$$

Для простоты изложения в дальнейшем будем предполагать $M\xi_0 = 0$. Это предположение не умаляет общности, но в то же самое время даёт возможность, отождествляя ковариацию со скалярным произведением, применять методы и результы теории гильбертовых пространств.

Обозначим $$R(n) = cov(\xi_{n}, \ \xi_{0}), \ n \in \mathbb {Z},$$ и (в предложении $R(0) = M|\xi_{0}|^2 \neq 0$) $$\rho(n) = R(n)/R(0), \ n \in \mathbb {Z}.$$

Функцию $R(n)$ будем называть ковариационной функцией, а $\rho(n)$ -- кореляционной функцией (стационарной в широком смысле) последовательности $\xi.$

Непосредственно из определения следует, что ковариационная функция $R(n)$ является неотрицательно-определённой, т.е. для любых комплексных чисел $a_{1}, \ ..., \ a_{m}$ и $t_{1}, \ ..., \ t_{m} \in \mathbb {Z}, \ m \geq 1$ 

$$\sum_{i,j=1}^m a_i \overline{a_j}R(t_i - t_j) \geq 0.$$

В свою очередь отсюда (или непосредственно из определения) нетрудно вывести следующие свойства ковариационной функции:
$$R(0) \geq 0, \ R(-n) = \overline{R(n)}, \ |R(n)| \leq R(0),$$
$$|R(n)-R(m)|^2 \leq 2R(0)[R(0) - Re \ R(n-m)](*).$$

\subsection{Доказательство $(*)$}

\emph{[М.Г. Широбоков ''Лекции по случайным процессам''. 2022 г., стр. 46 - 47]}

У стационарных процессов ковариационная функция является
функцией только разности компонент. Удобно поэтому ввести функцию

$$R_X (t) = R_X (t,0) = \mathbb{E}\overset{\circ}{X}(t) \overline{\overset{\circ}{X}(0)}$$ которую мы так же будем называть ковариационной функцией стационарного процесса X. 

Где $\overset{\circ}{X}$ - ''центрированная'' случайная величина $X$. То есть $\overset{\circ}{X} = X - \mathbb EX$

Эта функция будет определена для всех $t \in \mathbb{R}$, а не только для $t \in T = [0,\infty) $, если мы доопределим функцию $R_X (t,s) $ с помощью равенства $R_X(t,s) = R_X(t+h,s+h)$ для всех $h \in \mathbb{R}$.  Займемся теперь выводом свойства ковариационных функций произвольных
комплексных стационарных случайных процессов.

Учитывая, что из выражений

$$R_X(t,s) = R_X(t-s,0) = R_X(t-s),$$
$$R_X(t,s) = R_X(0,s-t) = \overline{ R_X(s-t,0)} = \overline {R_X(s-t)}$$

следует, что
$$R_X(t) = \overline{ R_X(-t)}, \ t \in \mathbb {R}.$$

Заметим, что
$$\mathbb{E}|X(t+h) - X(t)|^2 = \mathbb{E}|\overset{\circ}{X}(t+h) - \overset{\circ}{X}(t)|^2 = 2(R_X (0) - \mathrm{Re}\,R_{X}(h)).$$

Из неравенства Коши–Буняковского следует, что
$$|\mathbb{E}(\overset{\circ}{X}(t+h) - \overset{\circ}{X}(t)) \overline{\overset{\circ}{X}(t-s)}|^2 \leq \mathbb{E}|\overset{\circ}{X}(t+h) - \overset{\circ}{X}(t)|^2 \cdot \mathbb{E}|\overset{\circ}{X}(t-s)|^2,$$

что равносильно

$$|R_{X}(s+h)-R_{X}(s)|^{2}\leq2(R_{X}(0)-\mathrm{Re}\,R_{X}(h))R_{X}(0).$$

ч.т.д.

\subsection{Пример почти периодического процесса}

\emph{[А.Н. Ширяев ''Вероятность'', стр. 404 - 405]}

Приведём пример стационарной последовательности $\xi = (\xi_{n})_{n \in \mathbb {Z}}$. (Примечание: в дальнейшем слова ''в широком смысле'', а так же указание на то, что $n \in \mathbb {Z}$, часто будут опускаться.)

Пусть $$\xi_n = \sum_{k=1}^N z_k e^{i \lambda_{k}n}, \ \ \ (1)$$ где $z_1, \ ..., \ z_N $ -- ортогональные ($M_{z_i \overline{z}_j} = 0, \ i \neq j$) случайные величины с нулевыми средними и $M|z_k|^2 = \sigma^2_k > 0$; $-\pi \leq \lambda_k < \pi, \ k = 1, \ ..., N \ \lambda_i \neq \lambda_j, \ i \neq j$. Последовательность $\xi = (\xi_n)$ является стационарной с $$R(n) = \sum_{k=1}^N \sigma_k^2 e^{i \lambda_{k}n}.$$ 

В обобщение (1) предположим теперь, что $$\xi_n = \sum_{k= - \infty}^{+\infty}\sigma_k^2 e^{i \lambda_{k}n}, \ \ \ (2)$$ где величины $z_k, k \in \mathbb {Z}$ обладают теми же свойствами что и в (1). Если предположить, что $\sum_{k= - \infty}^{+\infty} \sigma_k^2 < \infty,$ то ряд в правой части формулы (2) сходятся в среднеквадратическом смысле и $$R(n) = \sum_{k= - \infty}^{+\infty}\sigma_k^2 e^{i \lambda_{k}n}. \ \ \ (3)$$

Введём функцию $$F(\lambda) = \sum_{k: \ \lambda_k \leq \lambda}^{} \sigma_k^2. \ \ \ (4)$$

Тогда ковариационная функция (3) может быть записана в виде интеграла Лебега--Стилтьеса:
$$R(n) = \int_{-\pi}^{\pi} e^{i\lambda n}dF(\lambda). \ \ \ (5)$$

Стационарные последовательности (2) образованы как суммы "гармоник" $\ e^{i \lambda_{k}n}$ с \\ "частотой" $\ \lambda_k$ и случайными "амплитудами" $\ z_k$ "интенсивности" $\ M|z_k|^2 = \sigma^2_k.$ Таким образом, знание функции $\ F(\lambda) \ $ даёт исчерпывающую информацию о структуре "спектра" последовательности $\ \xi,$ то есть о величине интенсивностей, с которыми те или иные частоты входят в представление (2). Согласно (5) значение функции $\ F(\lambda) \ $ полностью определяет также и структуру ковариационной функции $\ R(n).$

С точностью до постоянного множителя (невырожденная) функция $F(\lambda)$ является, очевидно, функцией распределения, причем в рассматриваемом примере эта функция кусочно-постоянна. Весьма примечательно, что ковариационная функция любой стационарной в широком смысле случайной последовательности может быть представлена в виде (5), где $F(\lambda)$ некоторая (с точностью до нормировки) функция распределения, носитель которой сосредоточен на множестве $[-\pi, \pi),$ т.е. $F(\lambda) = 0$ для $\lambda < -\pi$ и $F(\lambda) = F(\pi)$ для $\lambda > \pi.$

\subsection{Пример с непрерывной функцией}

\emph{[А.В. Гасников, Э.А. Горбунов, С.А. Гуз и др. ''Лекции по случайным процессам: учебное пособие'', 2019 г., стр 85 - 86]}

Дан случайный процесс $Z(t) = A cos (Bt + \phi), \  t \leq 0,$ в котором A, B и $\phi$ являются случайными величинами, причем $\phi$ не зависит от A и B и распределено равномерно на отрезке $[0, 2\pi]$. Про A и B известно, что они имеют совместную плотность распределения $f(a, b)$ и $A \geq 0, B \geq 0$ п.н. Исследуем процесс $Z(t)$ на стационарность в широком смысле.

Вычислим математическое ожидание:

$$\mathbb {E}Z(t) = \mathbb {E}A cos (Bt + \phi) = \mathbb {E}A cos (Bt) cos \phi - \mathbb {E}A sin (Bt)sin\phi.$$

Так как $\phi$ не зависит от A и B, то $\mathbb {E}A cos (Bt) cos \phi = \mathbb {E}A cos( Bt)·\mathbb {E}cos \phi= 0$
(в предположении, что $\mathbb {E}|A| < \infty$), что следует из того, что

$$\mathbb {E}cos \phi = \int_{0}^{2\pi} (1/2\pi) cos(x)dx.$$

Аналогично получаем, что $\mathbb {E}A sin( Bt)sin \phi = \mathbb {E}A sin( Bt·\mathbb {E}) sin \phi = 0.$ Следовательно, $\mathbb {E}Z(t) = 0$ для любых $t \geq 0$, т.е. от времени не зависит.

Вычислим теперь ковариационную функцию (в предположении, что $\mathbb {E}A^2 < \infty$)


$$R_Z(t, s) = \mathbb {E}Z(t)Z(s) - \underbrace{\mathbb {E}Z(t)\mathbb {E}Z(s)}_{0}$$

$$= \mathbb {E}A^2 cos (Bt + \phi)  cos(Bs + \phi)$$

$$= \mathbb {E}A^2 \cdot \frac{1}{2} \bigg( cos \bigg(\frac {B(t+s)}{2} + \phi \bigg)  cos \bigg(\frac {B(t+s)}{2} \bigg) =$$

$$= \underbrace{ \mathbb {E}A^2 \cdot \frac{1}{2} cos (\frac {B(t-s)}{2})}_{function \ t-s} + \underbrace{ \mathbb {E}A^2 \cdot \frac{1}{2}  cos \bigg(\frac {B(t+s)}{2} + \phi \bigg)}_{0}.$$


Получается, что ковариационная функция $R_{Z}(t, s)$ зависит от $t$ и $s$ только через их разность. Принимая во внимание постоянность математического ожидания, заключаем, что процесс $Z(t)$ является стационарным в широком смысле (в предположении, что $\mathbb {E}A^2 < \infty$). 

