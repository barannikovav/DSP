
 \section{Определение однородной Марковской цепи с конечным множеством состояний\\ как случайного процесса}
 
 \textbf{Автор:} Шлапак Мария Владимировна, Б-01-001
 
\begin{enumerate}
\item \textbf{Предварительные сведения.} Определим сначала несколько основных понятий, использующихся в курсе теории вероятности и теории случайных процессов.

Рассмотрим некоторый эксперимент, все мыслимые исходы которого описываются конечным числом различных исходов $\omega_1, ..., \omega_N$. Исходы $\omega_1, ..., \omega_N$ будем называть \textit{элементарными событиями}, а их совокупность
\begin{equation*}
\Omega = \{\omega_1, ..., \omega_N\}
\end{equation*}
(конечным) \textit{пространством элементарных событий} или \textit{пространством исходов}.

Будем обозначать все подмножества $A \subseteq \Omega$, для которых по условиям эксперимента возможен ответ одного из двух типов: <<исход $\omega \in A$>> или <<исход $\omega \notin A$>>, -- будем называть \textit{событиями}. Тогда множество $\Omega$ естественно назвать необходимым, или \textit{достоверным} событием. 

Если рассматривается некая система $\mathcal{A_{0}}$ множеств $A \subseteq \Omega$, то с помощью теоретико-множественных операций можно из элементов $\mathcal{A_0}$ построить новую систему множеств, которые также являются событиями. Присоединяя к этим событиям достоверное и невозможное события $\Omega$ и $\varnothing$, получаем систему множеств $\mathcal{A}$, которая является \textit{алгеброй}, т.е. такой системой подмножеств множества $\Omega$, что
\begin{enumerate}
\item $\Omega \in \mathcal{A}$,
\item если $A \in \mathcal{A}$, $B \in \mathcal{A}$, то множества $A\cup B$, $A \cap B$, $A \ B$ также принадлежат $\mathcal{A}$. 
\end{enumerate}

Будем говорить, что система множеств 
\begin{align*}
\mathcal{D} = \{D_1, ... , D_n\}
\end{align*}
образует \textit{разбиение} множества $\Omega$, а $D_i$ являются \textit{атомами} этого разбиения, если множества $D_i$ непусты, попарно не пересекаются и их сумма равна $\Omega$. 

Если рассматривать всевозможные объединения множеств из $\mathcal{D}$, то вместе с пустым множеством полученная система множеств будет алгеброй, которая называется \textit{алгеброй порожденной разбиением $\mathcal{D}$} и обозначается как $\alpha(\mathcal{D})$. Итак, если $\mathcal{D}$ -- некоторое разбиение, то ему однозначным образом ставится в соответствие алгебра $\mathcal{B} = \alpha(\mathcal{D})$.

Пусть $P = P(A)$ -- вероятностная мера, определенная на борелевских множествах $A$ числовой прямой. Возьмем $A = (-\infty, x]$, и положим
\begin{align*}
F(x) = P(-\infty, x], x \in R.
\end{align*}
Так определенная функция обладает следующими свойствами:
\begin{itemize}
\item $F(x)$ -- неубывающая функция;
\item $F(- \infty) = 0, F(+ \infty) = 1$, где 
\begin{align*}
F(- \infty) = \lim_{x \downarrow - \infty} F(x), F(+ \infty) = \lim_{x \uparrow + \infty} F(x);
\end{align*} 
\item $F(x)$ непрерывна справа и имеет пределы слева в каждой точке $x \in R$.
\end{itemize}
\begin{definition}\label{mariyka_def_1} Всякая функция $F = F(x)$, удовлетворяющая перечисленным условиям, называется \textit{функцией распределения} (на числовой прямой $R$).
	\end{definition}

\begin{definition}\label{mariyka_def_2} Система $\mathcal{F}$ подмножеств $\Omega$ называется \textit{$\sigma$-алгеброй}, если она является алгеброй и, кроме того, выполнено следующее свойство:
\begin{align*}
\text{если } A_n \in \mathcal{F}, n = 1,2,..., \text{то} \cup A_n \in \mathcal{F}, \cap A_n \in \mathcal{F}
\end{align*}
\end{definition}

\textbf{\textit{Основное определение.}} Набор объектов $(\Omega, \mathcal{F}, P)$,
\begin{itemize}
\item $\Omega$ -- множество точек $\omega$,
\item $\mathcal{F}$ -- $\sigma$-алгебра подмножества $\Omega$,
\item $P$ -- вероятность на $\mathcal{F}$,
\end{itemize}
называется \textit{вероятностной моделью} или \textit{вероятностным пространством}. При этом $\Omega$ называется \textit{пространством исходов} или \textit{пространством элементарных событий}, множества $A$ из $\mathcal{F}$ -- \textit{событиями}, а $P(A)$ -- \textit{вероятностью события} $A$.

\item \textbf{Случайные элементы}. Наряду со случайными величинами в теории вероятностей и ее приложениях рассматривают случайные объекты более общей природы, например случайные <<точки>>, векторы, функции, процессы, поля, множества, меры и т. д. В связи с этим желательно иметь понятие случайного объекта произвольной природы.

\begin{definition}\label{mariyka_def_3} Пусть $(\Omega, \mathcal{F})$ и $(E, \xi)$ -- два измеримых пространства. Будем говорить, что функция $X = X(\omega)$, определенная на $\Omega$ и принимающая значения в $E$, есть $\mathcal{F}|\xi$\textit{-измеримая функция}, или \textit{случайный элемент} (со значениями в $E$), если для любого $B \in \xi$ 
\begin{align*}
\{\omega: X(\omega) \in \text{B}\} \in \mathcal{F}.
\end{align*}
\end{definition}
Рассмотрим частные случаи этого определения.

Если $(E, \xi) = (R, \mathcal{B}(R))$, то определение случайного элемента совпадает с определением случайной величины. 

Пусть $(E, \xi) = (R^n, \mathcal{B}(R^n))$. Тогда случайный элемент $X(\omega)$ есть <<случайная точка>> в $R^n$. 

Пусть $(E, \xi) = (R^{T}, \mathcal{B}(R^T))$, где $T$ -- некоторое подмножество числовой прямой. В этом случае всякий случайный элемент $X = X(\omega)$, представимый в виде $X = (\xi_t)_{t \in T}$ называют случайной функцией с временным интервалом $T$. 

\begin{definition}\label{mariyka_def_4} Пусть $T$ -- некоторое подмножество числовой прямой. Совокупность случайных величин $X = (\xi_t)_{t \in T}$ называется \textit{случайным процессом с временным интервалом $T$}. Если $T = \{1,2, ...\}$, то $X = (\xi_1, \xi_2 , ...)$ называют \textit{случайным процесом с дискретным временем}. 
\end{definition}

\item \textbf{Построение процесса с заданными конечномерными распределениями}. Пусть $\xi = \xi(\omega)$ -- случайная величина, заданная на вероятностном пространстве $(\Omega, \mathcal{F}, P)$ и 
\begin{align*}
F_{\xi} (x) = P\{\omega: \xi(\omega) \leq x\} 
\end{align*}
-- её функция распределения. Тогда $F_{\xi} (x)$ является функцией распределения на числовой прямой в смысле определения \ref{mariyka_def_1}.

(!) Важное замечание состоит в том, что такая случайная величина действительно существует. 

Поставим аналогичный вопрос о существовании \textit{случайного процесса}. Пусть $X = (\xi_t)_{t \in T}$ -- случайный процесс (в смысле определения \ref{mariyka_def_4}), заданный на некотором вероятностном пространстве $(\Omega, \mathcal{F}, P)$, т.е. случайный процесс в узыком смысле, для $t \in T \subseteq R$. 

С физической точки зрения наиболее важной вероятностной характеристикой случайного процесса является набор $\{F_{t_1, ... , t_n} (x_1 , ..., x_n)\}$ его \textit{конечномерных функций распределения}
\begin{align*}
F_{t_1, ... , t_n} (x_1 , ..., x_n) = P\{\omega: \xi_{t_1} \leq x_1, ... ,\xi_{t_n} \leq x_n \},
\end{align*}
заданных для всех наборов $t_1, ..., t_n$ с $t_1 < t_2 < ... < t_n$.

Из данной формулы видно, что для каждого набора $t_1, ..., t_n$ с $t_1 < t_2 < ... < t_n$ функции $F_{t_1, ... , t_n} (x_1 , ..., x_n)$ являются n-мерными функциями распределения и что набор $\{ F_{t_1, ... , t_n} (x_1 , ..., x_n) \}$ удовлетворяет следующим \textit{условиям согласованности}:
\begin{align*}
F_{t_1, ... , t_n} (x_1 , ..., x_n) = F_{t_1, ... , t_n} (x_1 , ..., x_k, +\infty, ... , + \infty).
\end{align*}

Естественно теперь поставить такой вопрос: при каких условиях заданное семейство $\{ F_{t_1, ... , t_n} (x_1 , ..., x_n) \}$ функций распределения $F_{t_1, ... , t_n} (x_1 , ..., x_n)$ может быть семейством конечномерных функций распределения некоторого случайного процесса? 

Ответ на этот вопрос можно получить с помощью теоремы Колмогорова.


\begin{theorem}[теорема Колмогорова о существовании процесса]
Пусть $\{ F_{t_1, ... , t_n} (x_1 , ..., x_n) \}$, где $t_i \in T \subseteq R$, $t_1 < t_2 < ... < t_n$, $n\geq1$, -- заданное семейство конечномерных функций распределения, удовлетворяющих условиям согласованности. Тогда существуют вероятностное пространство $(\Omega, \mathcal{F}, P)$ и случайный процесс $X = (\xi_t)_{t \in T}$, такие, что
\begin{align*}
P\{\omega: \xi_{t_1} \leq x_1, ... ,\xi_{t_n} \leq x_n \} =  F_{t_1, ... , t_n} (x_1 , ..., x_n). 
\end{align*}
\end{theorem}

Сформулируем два важных следствия из этой теоремы:

\begin{corollary}\label{mariyka_cor_1} Пусть $F_1(x), F_2(x), ...$ -- последовательность одномерных функций распределения. Тогда существуют вероятностное пространство  $(\Omega, \mathcal{F}, P)$ и последовательность независимых случайных величин $xi_1, \xi_2 ,... $ такие, что
\begin{align*}
P\{ \omega: \xi_i(\omega) \leq x\} = F_i(x).
\end{align*}
\end{corollary}

\begin{corollary}\label{mariyka_cor_2} Пусть $T = [0, \infty)$ и $\{p(s, x; t, B)\}$ -- семейство неотрицательных функций, определенных для $s,t \in T, t> s, x \in R, B \in \mathcal{B(R)}$ и удовлетворяющих следующим условиям:
\begin{itemize}
\item $p(s, x; t, B)$ является при фиксированных $s,x,t$ \textit{вероятностной мерой по B};
\item при фиксированных $s,t,B$ $p(s, x; t, B)$ является \textit{борелевской функцией по x};
\item для всех $0 \leq s < t < \tau$ и $B \in \mathcal{B(R)}$ выполняется \textit{уравнение Колмогорова-Чэпмена}
\begin{align*}
p(s, x; \tau, B) = \int\limits_R p(s,x; t, dy)p(t,y;\tau, B).
\end{align*}
\end{itemize}
И пусть $\pi = \pi(B)$ -- вероятностная мера на $(R, \mathcal{B(R)})$. Тогда существуют вероятностное пространство  $(\Omega, \mathcal{F}, P)$ и случайный процесс $X = \{\xi_t\}_{t\geq 0}$ на нем такие, что для $0 = t_0 <t_1 < t_2 < ... < t_n$, 
\begin{align*}
P\{ \xi_{t_0} \leq x_0, ... ,\xi_{t_n} \leq x_n \} = \int\limits_{-\infty}^{x_0}\pi(dy_0) \int\limits_{-\infty}^{x_1}p(0, y_0; t_1, dy_1)...\int\limits_{-\infty}^{x_n}p(t_{n-1}, y_{n-1}; t_n, dy_n).
\end{align*}
\end{corollary}
Так построенный процесс называется \textbf{\textit{марковским процессом}} с начальным распределением $\pi$ и системой переходных вероятностей $\{p(s,x;t,B)\}$.

\item \textbf{Марковская цепь как случайный процесс}.

Положим для \textit{следствия \ref{mariyka_cor_2}} $T = \mathcal{N}$ и сделаем пространство состояний дискретным. Поскольку все остальные условия \textit{следствия \ref{mariyka_cor_2}} остались выполнены, то для предложенных вероятностей существует пространство $(\Omega, \mathcal{F}, P)$ и случайный процесс, который в этом случае (случае дискретного времени и пространства состояний) называется \textbf{цепью Маркова}.

\end{enumerate}  

