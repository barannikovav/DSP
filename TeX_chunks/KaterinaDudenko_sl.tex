
\section{Статистическая оценка математического ожидания гауссовской стационарной последовательности}

 \textbf{Автор:} Дуденко Екатерина Игоревна, Б-01-001

\subsection{Гауссовский случайный процесс.}
\begin{definition}(Лекции по случайным процессам А.В. Гасников \cite{GasnikovLectionsRP})
Гауссовский процесс - это процесс, все конечномерные распре-
деления которого нормальные (гауссовские). Это значит, что любой
случайный вектор, составленный из сечений такого процесса, имеет
нормальное распределение.
\end{definition}

Напомним, что случайный вектор $X \in \mathbb{\bf{R}} ^ n$ является гауссовским с
математическим ожиданием $m$ и корреляционной матрицей $R \in \mathbb{\bf{R}} ^{n \times n}$
(обозначается $X \in \mathbb{N}(m, R)$), если его характеристическая функция за-
дается формулой:

\begin{equation*}
    \varphi_X(s)  \stackrel{def}{=} \mathbb{E}e^{i\langle{s,X}\rangle} = exp(i\langle{s,m}\rangle - \frac{1}{2}\langle{s,Rs}\rangle), s \in \mathbb{\bf{R}}^n
\end{equation*}

Если $det R \neq 0$ , то $X$ обладает плотностью распределения:
\begin{equation*}
    p_X(u) = \frac{1}{\sqrt{(2\pi)^n det R}} exp(-\frac{1}{2}\langle{u,R^{-1}u}\rangle), u \in \mathbb{\bf{R}^n}
\end{equation*}

\subsection{Стационарные последовательности.}

\begin{definition}(Ширяев вероятность \cite{ShiryaevVeroyatnost1})
Случайная последовательность $\xi = (\xi_1, \xi_2, ...)$ называется стационарной в узком смысле, если для любого множества $B \in \mathbb{\bf{B}}(\mathbb{\bf{R}}^{\infty})$ и любого $n \geq 1$
\begin{equation}
\mathbf{P}\{(\xi_1, \xi_2, ...) \in B \} = \mathbf{P}\{ (\xi_{n+1}, \xi_{n+2}, ...) \in B \}
\end{equation}
\end{definition}
Отсюда, в частности, вытекает, что если $\mathbf{M}\xi_1^2 < \infty$, то $\mathbf{M}\xi_n$ не зависит от $n$: $\mathbf{M}\xi_n=\mathbf{M}\xi_1$

\begin{definition}(Ширяев вероятность \cite{ShiryaevVeroyatnost1})
Последовательность комплексных случайных величин $\xi = (\xi_n)_{n\in\mathbb{Z}}$ с $\mathbf{M}|\xi_n|^2 < \infty, n\in\mathbb{Z}$, называется стационарной (в широком смысле), если для всех $n \in \mathbb{\bf{Z}}$ 

\begin{equation}
\begin{gathered}
\mathbf{M}\xi_n = \mathbf{M}\xi_0 \\
cov(\xi_{k+n}, \xi_k) = cov(\xi_n, \xi_0), \  k\in\mathbb{\bf{Z}}.
\end{gathered}
\end{equation}

\end{definition}

\begin{theorem}(Лекции по случайным процессам А.В. Гасников \cite{GasnikovLectionsRP}) 
Гауссовский процесс является стационарным в широком смысле тогда и только тогда, когда он является стационарным в узком смысле.
\end{theorem}

\subsection{Спектральное представление стационарных последовательностей.}

Если $\xi = (\xi_n)$ - стационарная последовательность с $\mathbf{M}\xi_n = 0, n \in \mathbb{\bf{Z}},$ то, найдется такая конечная мера $F = F(\delta)$ на, $([-\pi, \pi), \mathscr{B}([-\pi, \pi))),$ что ковариационнная функция $R(n) = cov(\xi_{k+n}, \xi_k)$ допускает спектральное представление
\begin{equation}
R(n) = \int_{-\pi}^{\pi} e^{i\lambda n}F(d\lambda).
\end{equation}
Следующий результат дает соответствующее спектральное представление самой последовательности.
\begin{equation}\label{spkt}
\xi_n = \int_{-\pi}^{\pi} e^{i\lambda n}Z(d\lambda).
\end{equation}

При этом $\mathbf{M}|Z(\delta)|^2 = F(\delta)$.

\begin{theorem}\label{t1}(Ширяев вероятность \cite{ShiryaevVeroyatnost1})
Пусть $\xi = (\xi_n)\, n \in \mathbb{Z}\,$ стационарная последовательность с $\mathbf{M}\xi_n = 0$, ковариационной функцией $R(n) = \int_{-\pi}^{\pi}   e^{i\lambda n}F(d\lambda)$ и спектральным разложением $\xi_n = \int_{-\pi}^{\pi} e^{i\lambda n)Z(d\lambda}$. Тогда:
\begin{equation}\label{eq1}
\frac{1}{n}\sum_{k=0}^{n-1}\xi_k \xrightarrow{L^2}Z(\{ 0\})
\end{equation}
$$
\text{и}
$$
\begin{equation}\label{eq2}
\frac{1}{n}\sum_{k=0}^{n-1} R(k) \xrightarrow{}F(\{ 0\})
\end{equation}
\end{theorem}



\begin{proof} 

В силу (\ref{spkt})


\[	
\frac{1}{n} \sum_{k=0}^{n-1}\xi_k = \int_{-\pi^\pi}\sum_{k=0}^{n-1}e^{ik\lambda}Z(d\lambda) = \int_{-\pi}^{\pi}\varphi_n(\lambda)Z(\lambda), \\
\]
\[
\text{где } \varphi_n(\lambda) = \frac{1}{n}\sum_{k=0}^{n-1}e^{ik\lambda} =
\begin{cases*}
1, \lambda = 0,\\
\frac{1}{n}*\frac{e^{in\lambda} - 1}{e^{i\lambda} - 1}, \lambda \neq 0.
\end{cases*}
\]
$$
\text{Поскольку } |\sin\lambda| \geq \frac{2}{\pi}|\lambda| \text{ для } |\lambda|\leq\frac{\pi}{2}, \text{ то}
$$
$$
 |\varphi_n(\lambda)|= \Bigg|\frac{\sin{\frac{n\lambda}{2}}}{n\sin{\frac{\lambda}{2}}}\Bigg| \leq \frac{\pi}{2}
\Bigg|\frac{\sin{\frac{n\lambda}{2}}}{\frac{n\lambda}{2}}\Bigg| \leq \frac{\pi}{2}.
$$
$$
\text{Далее, } \varphi_n(\lambda) \xrightarrow{L^2} I_{\{0\}}(\lambda), \text{ поэтому } 
$$
$$
\int_{-\pi}^{\pi}\varphi_n(\lambda)Z(d\lambda) \xrightarrow{L^2} \int_{-\pi}^{\pi}  I_{\{0\}}(\lambda) Z(d\lambda) = Z({0}),
$$
$$
\text{что и доказывает (\ref{eq1}). Аналогичным образом доказывается и утверждение (\ref{eq2})}
$$
\end{proof}



\begin{corollary}
Если спектральная функция непрерывна в нуле, т. е. $F(\{0\}) = 0$, то $Z(\{0\}) = 0$ и в силу (\ref{eq1}), (\ref{eq2})
\begin{equation}
\frac{1}{n}\sum_{k=0}^{n-1} R(k) \xrightarrow{}0 \Rightarrow \frac{1}{n}\sum_{k=0}^{n-1}\xi_k(k) \xrightarrow{L^2}0 
\end{equation}
\end{corollary}

\begin{proof} 

Поскольку
\begin{equation*}
\Bigg| \frac{1}{n}\sum_{k=0}^{n-1}R(k)\Bigg|^2 = \Bigg|\mathbf{M}\Big(\frac{1}{n}\sum_{k=0}^{n-1}\xi_k\Big) \xi_0\Bigg|^2 \leq \mathbf{M}|\xi_0|^2\mathbf{M}\Big( \frac{1}{n}\sum_{k=0}^{n-1}\xi_k\Big)^2,
\end{equation*}
То верна и обратная импликация:
\begin{equation*}
\frac{1}{n}\sum_{k=0}^{n-1}\xi_k \xrightarrow{L^2}0 \Rightarrow \frac{1}{n}\sum_{k=0}^{n-1}R(k) \xrightarrow {} 0.
\end{equation*}
Таким образом, условие $\frac{1}{n}\sum_{k=0}^{n-1} R(k) \xrightarrow{}0$ является необходимым и достаточным для сходимости (в среднеквадратичном смысле) средних арифметических $\frac{1}{n}\sum_{k=0}^{n-1}\xi_k(k)$ к нулю. 

\end{proof}


\begin{corollary}
Отсюда следует, что для стационарной в широком смысле последовательности $\xi = (\xi_n)$ такой, что ее математическое ожидание есть $m$ $(\mathbf{M}\xi_0=m)$, то
\begin{equation}
\frac{1}{n}\sum_{k=0}^{n-1} R(k) \xrightarrow{}0 \Leftrightarrow \frac{1}{n}\sum_{k=0}^{n-1}\xi_k(k) \xrightarrow{L^2}m.
\end{equation}
где $R(n) = \mathbf{M}(\xi_n - \mathbf{M}\xi_n)(\xi_0 - \mathbf{M}\xi_0)$.
\end{corollary}

\subsection{Статистическое оценивание.}
Пусть $\xi = (\xi_n)$, $n \in \mathbf{\bf{Z}}$ стационарная в широком смысле случайная последовательность с математическим ожиданием $\mathbf{M}\xi_n = m$ и ковариацией $R(n) = \int_{-\pi}^{\pi} e^{i\lambda n}F(d\lambda)$.

Пусть $x_0, x_1, ..., x_{N-1}$ - полученные в ходе наблюдения случайных величин $\xi_0, \xi_1, ..., \xi_{N-1}$. По ним нужно построить 'хорошую' оценку (неизвестного) среднего значения $m$.

Положим
\begin{equation}
m_N(X) = \frac{1}{N}\sum_{k=0}^{N-1}x_k.
\end{equation}

Тогда из элементарных свойств математического ожидания следует, что эта оценка является 'хорошей' оценкой величины $m$ в том смысле, что 'в среднем по всем реализациям $x_0, ..., x_{N-1}$'  она является $несмещенной$, т. е

\begin{equation}
\mathbf{M}m_N(\xi) = \mathbf{M}(\frac{1}{N}\sum_{k=0}^{N-1}\xi_k) = m.
\end{equation}

Более того из теоремы \ref{t1} вытекает, что при условии $\frac{1}{N}\sum_{k=0}^{N-1} R(k) \xrightarrow{} 0, N \xrightarrow{} \infty$, рассматриваемая оценка является также и {\em состоятельной} (в среднеквадратичном смысле), т. е.

\begin{equation}
\mathbf{M}|m_N(\xi) - m|^2 \xrightarrow{} 0, N \xrightarrow{} \infty.
\end{equation}
Таким образом, получено требуемое.
