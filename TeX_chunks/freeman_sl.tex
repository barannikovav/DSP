


\section{Теорема Колмогорова о существовании случайного процесса, о согласованных распределениях.}

\textbf{Автор:} Пяточкин Вадим Сергеевич, Б-01-005

\subsection{Введение}

При изучении явлений окружающего мира человечество часто сталкивается с процессами, течение которых заранее предсказать в точности невозможно. Эта неопределенность (непредсказуемость) вызвана влиянием случайных факторов, воздействующих на ход процесса.
 
Примеры случайных процессов:

1. Напряжение в электросети, номинально постоянное и равное 220 B, фактически меняется во времени, колеблется вокруг номинала под влиянием таких случайных факторов, как количество и вид включенных в сеть приборов, моменты их включений и выключений и т. д.

2. Население города (или области) меняется с течением времени случайным (непредсказуемым) образом под влиянием таких факторов, как рождаемость, смертность, миграция и т. д.

3. Уровень воды в реке (или в водохранилище) меняется во времени случайным образом в зависимости от погоды, количества осадков, таяния снега, интенсивности оросительных мероприятий и т. д.

4. Частица, совершающая броуновское движение в поле зрения микроскопа, меняет свое положение случайным образом в результате соударений с молекулами жидкости.

5. Представим полет космической ракеты, которую необходимо вывести в заданный момент в заданную точку пространства с заданными направлением и абсолютным значением вектора скорости. Фактическое движение ракеты не совпадает с расчетным из-за таких случайных факторов, как турбулентность атмосферы, неоднородность горючего, ошибки в отработке команд и т. д.

Случайный процесс, протекающий в любой физической системе $S$, представляет собой случайные переходы системы из состояния в состояние. Состояние системы может быть охарактеризовано с помощью каких-то численных переменных: в простейшем случае - одной, а в более сложных - нескольких. В примере 1 процесс описывается одной переменной (напряжением $U$), случайным образом меняющейся во времени, являющейся функцией времени $U(t)$. Аналогично, в примере 2 население $N$ меняется случайным образом во времени: $N(t)$. Так же и в примере 3 случайный процесс характеризуется одной функцией $H(t)$, где $H$ - уровень воды в реке. Все эти три функции являются случайными функциями времени $t$. При фиксированном $t$ каждая из них превращается в обычную случайную величину. В результате опыта (когда он уже произведен) случайная функция превращается в обычную неслучайную функцию. Например, если в ходе времени непрерывно измерять напряжение в сети, получится неслучайная функция $u(t)$, колеблющаяся вокруг номинала $W_0$.

Несколько сложнее обстоит дело в примере 4: состояние частицы характеризуется уже не одной, а двумя случайными функциями $X(t)$ и $Y(t)$ – координатами частицы в поле микроскопа. Такой случайный процесс называется векторным, он описывается переменным случайным вектором, составляющие которого $X(t)$, $Y(t)$ меняются с течением времени. Для фиксированного значения аргумента случайный процесс превращается в систему двух случайных величин $X(t), Y(t),$ изображаемую случайной точкой (случайным вектором $Q(t)$) на плоскости $xOy$. При изменении аргумента $t$ точка $Q(t)$ будет перемещаться («блуждать») по плоскости. Еще сложнее обстоит дело с примером
5. Состояние ракеты в момент времени $t$ характеризуется не только
тремя координатами $X(t), Y(t), Z(t)$ центра массы ракеты, но и тремя составляющими ее скорости, тремя углами ориентации ракеты, угловыми скоростями движения вокруг центра массы, запасом топлива и т.п. Здесь пример многомерного случайного процесса: блуждание точки, описывающей состояние системы в момент времени $t$, происходит в многомерном пространстве. Сложности, связанные с изучением таких процессов, с увеличением размерности растут в огромной степени.

\subsection{Случайный процесс}

Случайная величина является одним из ключевых понятий в теории вероятностей и статистике. Она представляет собой величину, которая принимает различные значения в результате случайного эксперимента. 

Формально, случайная величина определяется как функция, которая отображает каждый элемент пространства элементарных исходов эксперимента в численное значение. Иными словами, она связывает возможные исходы с числами, которые представляют некоторую характеристику этого исхода.

Пусть $\xi = \xi(\omega)$ $-$ случайная величина, заданная на вероятностном пространстве ($\Omega$, $\mathscr{F}$, P) и 
$$F_\xi(x) = \textsc{P}\{ \omega: \xi(\omega) \leq x\} $$
$-$ ее функция распределения. Тогда существуют вероятностное пространство \\
($\Omega$, $\mathscr{F}$, P) и случайная величина $\xi = \xi(\omega)$ на нем такие, что
$$\textsc{P}\{ \omega: \xi(\omega) \leq x\} = F(x).$$

Случайный процесс(в узком смысле) - это семейство случайных величин на общем ($\Omega$, $\mathscr{F}$, P). Каждая случайная величина является результатом некоторого случайного эксперимента, который происходит в некоторый момент времени. Иными словами:

Пусть $X=(\xi_t)_{t \in T}$ $-$ случайный процесс, заданный на вероятностном пространстве ($\Omega$, $\mathscr{F}$, P) для $t \in T \subseteq R$. С физической точки зрения наиболее важной вероятностной характеристикой случайного процесса является набор 
\begin{equation} \label{link_1}
F_{t_1, ... , t_n}(x_1, ... , x_n) = P\{\omega: \xi_{t_1} \leq x_1, ... ,\xi_{t_n} \leq x_n\}, 
\end{equation}
заданных для всех наборов $t_1, ... , t_n$ с $t_1 < t_2 < ... < t_n$.

Из (\ref{link_1}) видно, что для каждого набора $t_1, ..., t_n$ с $t_1 < t_2 < ... < t_n$ функции  $F_{t_1, ... , t_n}(x_1, ... , x_n)$ является $n$-мерными функциями распределения и что набор \\ $\{F_{t_1, ... , t_n}(x_1, ... , x_n)\}$ удовлетворяет следующим условиям \emph{согласованности}:
\begin{equation} \label{link_2}
F_{t_1, ... , t_n}(x_1, ... , x_k) = F_{t_1, ... , t_n}(x_1, ... , x_k, +\infty, ... , +\infty), 
\end{equation}
где $k < n$.

Согласованные распределения - это совместное распределение нескольких случайных величин, в которых каждая случайная величина описывает одно и то же случайное явление, происходящее в различные моменты времени. То есть, согласованные распределения отвечают случайному процессу, если каждая случайная величина является элементом этого процесса и отвечает за некоторое событие, происходящее в определенный момент времени. 

Естественно теперь поставить такой вопрос: при каких условиях заданное семейство $\{F_{t_1, ... , t_n}(x_1, ... , x_n)\}$ функций распределния $F_{t_1, ... , t_n}(x_1, ... , x_n)$ может быть семейством конечномерных функций распределения некоторого случайного процесса в узком смысле? Весьма примечательно, что все такие дополнительные условия исчерпываются условиями согласованности (\ref{link_2})(в широком смысле).

\subsection{Теорема Колмогорова о существовании процесса}

\begin{theorem}
	Пусть $\{ F_{t_1, ... , t_n}(x_1, ... , x_n) \}$, где $t_i \in T \subseteq R$, $t_1 < t_2 < ... < t_n$, $n \geq 1$, заданное семейство конечномерных функций распределния, удовлетворяющих условиям согласованности (\ref{link_2}). Тогда существуют вероятностное пространство \\($\Omega$, $\mathscr{F}$, P) $=E=E_T$ и случайный процесс в узком смысле $X=(\xi_t)_{t \in T}$ на этом пространстве, такие что
	\begin{equation} \label{link_3}
	P\{\omega: \xi_{t_1} \leq x_1, ... ,\xi_{t_n} \leq x_n\} = F_{t_1, ... , t_n}(x_1, ... , x_k).
	\end{equation}
	Другими словами, Колмогоров доказал, что любой случайный процесс в широком смысле имеет модель в узком смысле.
\end{theorem}

\begin{proof} Положим
$$\Omega = R^T, \ \ \  \mathscr{F} = \mathscr{B}(R^T),$$
т.е. возьмем в качестве пространства $\Omega$ пространство действительных функций $\omega = (\omega_t)_{t \in T}$ с $\sigma$-алгеброй, порожденной цилиндрическими множествами вида $C_{t_1,...,t_n;x_1,...,x_n} = \ =\{ \omega \in \Omega| \ \omega_{t_1} \leq x_1, ... , \omega_{t_n} \leq x_n\} $.

Пусть $\tau = \{t_1, ... , t_n\}$, $t_1 < t_2 < ... < t_n$. Тогда, в пространстве $E_\tau=(R^\tau, \mathscr{B}(R^\tau))$ можно построить и при том единственную вероятностную меру $P_\tau$ такую, что
\begin{equation} \label{link_4}
P_\tau \{(\omega_{t_1}, ... , \omega_{t_n}): \omega_{t_1} \leq x_1, ... , \omega_{t_n} \leq x_n \} = F_{t_1, ... , t_n}(x_1, ... , x_k).
\end{equation}
Из условий согласованности (\ref{link_2}) вытекает, что семейство ${P_\tau}$ также является согласованным. На пространстве $E$ существует $\sigma$-аддетивная вероятностная мера P такая, что 
$$\textsc{P}\{\omega: (\omega_{t_1}, ... , \omega_{t_n}) \in B\} = P_\tau(B)$$
для всякого набора $\tau = \{_1, ... , t_n\}$, $t_1 < t_2 < ... < t_n$ и для $\forall B \in \mathscr{B}(R^T)$. Доказать счетность такой меры и есть наиболее трудная часть теоремы Колмогорова(без доказательства).

Отсюда следует также, что выполнено условие (\ref{link_4}). Таким образом, в качестве искомого случайного процесса $X=(\xi_t)_{t \in T}$ можно взять процесс, определенный следующим образом:
\begin{equation} \label{link_5}
\xi_t(\omega) = \omega_t, \ t \in T.
\end{equation}

\end{proof}

\begin{remark} Построенное вероятностное пространство $(R^T, \mathscr{B}(R^T), \textsc{P})$ часто называют каноническим, а задание случайного процесса равенством (\ref{link_5}) $-$ координатным способом построения процесса.
\end{remark}

\begin{corollary} Пусть $F_1(x), F_2(x), ...$  $-$ последовательность одномерных функций распределения. Тогда существуют вероятностное пространство ($\Omega$, $\mathscr{F}$, P) и последовательность независимых случайных величин $\xi_1, \xi_2, ...$ такие, что
\begin{equation}\label{link_6}
\textsc{P}\{ \omega: \xi_i(\omega) \leq x\} = F_i(x).
\end{equation}

В частности, существует вероятностное пространство ($\Omega$, $\mathscr{F}$, P), на котором определена бесконечная последовательность бернуллиевских случайных величин. Отметим, что в качестве $\Omega$ можно здесь взять пространство 
$$\Omega = \{ \omega: \omega=(a_1, a_2, ...), a_i = 0,1 \}$$
\end{corollary}

Для доказательства следствия достаточно положить $F_{1, ... , n}(x_1, ... , x_n) = F_1(x_1) ... F_n(x_n)$  и применить теорему Колмогорова.

\begin{corollary} Пусть $T = [0, \infty)$ и $\{p(s, x; t, B)\}$ $-$ семейство неотрицательных функций, определнных для $s, t \in T$, $t > s$, $x \in R$, $B \in \mathscr{B}(R)$ и удовлетворяющих следующим условиям:\\
a) $p(s, x; t, B)$ является при фиксированных $s, x$ и $t$ вероятностной мерой по $B$;\\
b) при фиксированных $s, t$ и $B$ $p(s, x; t, B)$ является борелевской функцией по $x$;\\
c) для всех $0 \leq s < t < \tau$ и $B \in \mathscr{B}(R)$ выполняется уравнение Колмогора-Чэпмена
$$p(s, x; \tau, B) = \int\limits_R p(s, x; t, dy)p(t, y; \tau, B).$$

И пусть $\pi = \pi(B)$ $-$ вероятностная мера на $(R, \mathscr(B)(R)).$ Тогда существуют вероятностное пространство ($\Omega$, $\mathscr{F}$, P) и случайный процесс $X = \{\xi_t\}_{t\geq 0}$ на нем такие, что для $0 = t_0 < t_1 < ... < t_n, \\  \textsc{P}\{\xi_{t_0} \leq x_0, \xi_{t_1} \leq x_1, ... , \xi_{t_n} \leq x_n \} = \int\limits_{-\infty}^{x_0} \pi(dy_0) \int\limits_{-\infty}^{x_1} p(0, y_0; t_1, dy_1)...\int\limits_{-\infty}^{x_n} p(t_{n-1}, y_{n-1}; t_n, dy_n)$

Так построенный процесс $X$ называется марковским процессом с начальным распределением $\pi$ и системой переходных вероятностей$\{ p(s, x; t, B)\}$
\end{corollary}

\subsection{Примеры применения}

Теорема Колмогорова является одной из ключевых теорем в теории вероятностей. Эта теорема позволяет описать свойства случайных величин, являющихся результатом совместного действия множества случайных факторов.

Применение теоремы Колмогорова в математической статистике:

1. Описательная статистика - используя теорему Колмогорова, можно определить распределение выборки и провести проверку гипотезы о том, что выборка была взята из конкретного распределения. Таким образом, теорема Колмогорова помогает в изучении случайных величин, которые могут быть полезны для выявления закономерностей в данных.

2. Метод максимального правдоподобия - теорема Колмогорова используется для определения параметров распределений на основе имеющихся данных.

Применение теоремы Колмогорова в финансовой математике:

1. Моделирование финансовых рынков - теорема Колмогорова применяется для моделирования случайной динамики цен на финансовых рынках. Таким образом, теорема Колмогорова помогает в определении вероятности изменения цены актива или ценности портфеля.

2. Опционное ценообразование - теорема Колмогорова учитывается при разработке моделей опционного ценообразования. Она является базовым инструментом для оценки рисков и определения цены опционов на финансовом рынке. 

Таким образом, теорема Колмогорова является важным инструментом для анализа случайных величин и их распределений в математической статистике и финансовой математике.
