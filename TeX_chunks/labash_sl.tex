
\section{Белый шум с дискретным временем}	

\textbf{Автор:} Лабашкин Глеб Александрович, Б-01-007

\subsection*{1. Нормальный случайный вектор} 

\begin{definition} Невырожденным нормальным (гауссовским) случайным вектором  $\vec{\xi} = (\xi_1, \xi_2, \dots, \xi_n)^T$
называется случайный вектор, обладающий плотностью распределения вероятностей вида:
$$\rho_{\vec{\xi}}(x_1, x_2, \dots, x_n) = \rho_{\vec{\xi}}(\vec{x}) = 
k \exp(\displaystyle\frac{1}{2}(\vec{x} - \vec{b})^T A (\vec{x} - \vec{b})),$$
где $A$ -- симметричная положительно определённая матрица,
$k$ -- нормировочный коэффициент, $b$ -- неслучайный вектор. \cite{GasnikovLectionsRP} 
\end{definition}

Осуществим некоторые преобразования. Известно, что для каждой симметричной матрицы $A$
существует ортогональная матрица $C$, такая, что: $$ C^T A C = D = \mathrm{diag}(d_1, d_2, \dots, d_n) ,$$
где $d_i > 0 ,~ i = \overline{1, n}$ в силу положительной определённости матрицы $A$. 
Введём матрицу \break $B = C D^{-1/2}$ и вектор $\vec{z}$,
удовлетворяющий преобразованию: $$\vec{x} - \vec{b} = B \vec{z} ,$$
$$ D^{-1/2} = \mathrm{diag}(\frac{1}{\sqrt{d_1}}, \frac{1}{\sqrt{d_2}}, \dots, \frac{1}{\sqrt{d_n}}) .$$
C учётом данного преобразования:
$$ (\vec{x} - \vec{b})^T A (\vec{x} - \vec{b}) =  \vec{z}^T B^T A B \vec{z} = 
\vec{z}^T D^{-1/2} C^T A C D^{-1/2} \vec{z} = \vec{z}^T \vec{z}.$$
Якобиан преобразования:
$$J(\vec{x}, \vec{z}) = \left\lvert \frac{\partial{x_i}}{\partial{z_j}} \right\rvert = \det(B).$$
Запишем плотность распределения $f_{\vec{\xi}}$ случайного вектора $\vec{\xi}$ в переменных $\vec{z}$:
$$ f_{\vec{\xi}}(z_1, z_2, \dots, z_n) = f_{\vec{\xi}}(\vec{z}) =
k \det(B) \exp(-\frac{1}{2} \vec{z}^T \vec{z}) = k \det(B) \exp(-\frac{1}{2} \sum_{i = 1}^{n} z_i^2) .$$
Используя интеграл Пуассона $\displaystyle\int_{-\infty}^{+\infty} e^{-t^2} dt = \sqrt{\pi}$ и условие нормировки
$\displaystyle\int_{\mathbb{R}^n} f_{\vec{\xi}}(\vec{z}) d\vec{z} = 1$, выразим коэффициент $k$:
$$ \int_{\mathbb{R}^n} k \det(B) \exp(-\frac{1}{2} \vec{z}^T \vec{z}) d\vec{z} = 
(\sqrt{2\pi})^n k \det(B) = 1 ~ \Rightarrow ~ k = \frac{1}{(\sqrt{2\pi})^n \det(B)} .$$
Таким образом, плотность распределения принимает вид:
$$ f_{\vec{\xi}}(\vec{z}) = \frac{1}{(\sqrt{2\pi})^n} \exp(-\frac{1}{2} \sum_{i = 1}^{n} z_i^2) .$$
Из этой формулы мы можем видеть, что соответсвующий ранее данному определению нормальный случайный вектор
посредством линейной замены сводится к случайному вектору,
каждая из компонент которого имеет стандартное нормальное распределение: \\\\
$$\xi_i \in N(0, 1) ,~ f_{\xi_i}(z_i) = \frac{1}{\sqrt{2\pi}} \exp(-\frac{1}{2} z_i^2) ,~ i = \overline{1, n} ,$$
$$ f_{\vec{\xi}}(\vec{z}) = f_{\xi_1}(z_1) f_{\xi_2}(z_2) \dots f_{\xi_n}(z_n) .$$ 

\begin{remark} Обратное в общем случае неверно. \\
Если каждая из компонент случайного вектора имеет нормальное распределение, это ещё не означает,
что сам вектор будет гауссовским. Однако, это справедливо для случая,
когда компоненты вектора независимы в совокупности. \\\\
\end{remark}


\subsection*{2. Белый шум с дискретным временем}

\begin{definition} Нормальным (гауссовским) случайным процессом называется случайный процесс $\xi(t), t \in T$,
все конечномерные распределения которого нормальны. Это означает, что любой случайный вектор вида
$(\xi(t_1), \xi(t_2), \dots, \xi(t_n))^T ,~ n \in \mathbb{N} ,~ t_i \in T ,~ i = \overline{1, n}$
является нормальным (гауссовским) случайным вектором. \cite{NatanTeorVero2007} \\
\end{definition}

Рассматриваем множество $T \in \{ \mathbb{N}, \mathbb{N}_0 = \mathbb{N} \cup \{0\} \}$.
Пусть $\mathscr{P}_{fin}(T) =  \{ \mathfrak{t} \subset T \mid Card(\mathfrak{t}) \in \mathbb{N} \}$,
где $Card(\mathfrak{t})$ -- кардинальное число (или же мощность) множества $\mathfrak{t}$.
Это означает, что $\mathscr{P}_{fin}(T)$ является системой
всех конечных непустых поднможеств $\mathfrak{t}$ множества $T$.
Тогда элементы любого такого подмножества $\mathfrak{t} \in \mathscr{P}_{fin}(T)$ могут быть пронумерованы:
$\mathfrak{t} = \{ t_1 < t_2 < \dots < t_{Card(\mathfrak{t})} \}$.
$t_i \in \mathfrak{t}, i = \overline{1, Card(\mathfrak{t})}$ представляют собой моменты времени,
в которые осуществляется измерение некоторого физического процесса, создающего белый шум.
Для ясности изложения, можно говорить, что мы осуществляем запись шума водопада (который довольно близок к белому).
Измеряемой случайной величиной $\xi_i \equiv \xi_{t_i} \equiv \xi(t_i)$ в таком случае будет
отклонение мембраны микрофона записывающего устройства в момент времени $t_i$.
С целью получения нормального процесса, положим, что $\xi_i$ распределены по Гауссу
(хоть в контексте нашего примера это и нефизично) и независимы в совокупности:
$$ \rho_{\xi_i}(x) = \frac{1}{\sqrt{2\pi}} e^{-x^2/2} ,~ i = \overline{1, Card(\mathfrak{t})} .$$

\begin{definition} Случайный процесс $\xi(t), t \in T$,
удовлетворяющий сформулированным выше условиям, будем называть гауссовским белым шумом с дискретным временем. \cite{ShamarovDRP} \\
\end{definition}

\noindent Тогда не составляет труда выписать вероятностную меру для процесса:
$$\forall B \in \mathscr{B}(\mathbb{R}^{\mathfrak{t}}) ~ \mathbb{P}_{\xi}^{\mathfrak{t}}(B) \equiv 
\mathbb{P}_{\xi_1, \xi_2, \dots, \xi_{Card(\mathfrak{t})}}(B) = 
\int_{\vec{x} \in B} \frac {exp(-\Vert \vec{x} \Vert ^2 / 2)} {(\sqrt{2\pi}) ^ {Card(\mathfrak{t})}}
Leb_{\mathbb{R}^{\mathfrak{t}}}(d\vec{x}) ,$$ 
где $\mathscr{B}(\mathbb{R}^{\mathfrak{t}})$ -- борелевская сигма-алгебра на $\mathbb{R}^{\mathfrak{t}}$, 
$Leb_{\mathbb{R}^{\mathfrak{t}}}$ -- мера Лебега на $\mathbb{R}^{\mathfrak{t}}$,
$\Vert \vec{x} \Vert ^2 = \sum_{i = 1}^{Card(\mathfrak{t})} x_i^2$. \\

\begin{remark} Для построения случайного процесса белого шума в качестве маргинальных распределний
$\xi(t_i)$ вовсе не обязательно было полагать нормальное распеределние. Как известно из физики, белый шум характерен
равномерным распределением его спектральных составляющих. Это достигается не за счёт конкретного вида распределения
процесса $\xi$, а благодаря независимости (а если быть строгим, некоррелированности) значений,
измеряемых в разные моменты времени $t_i \neq t_j$, что будет продемонстрировано в следующем разделе. \\\\
\end{remark}


\subsection*{3. Белый шум в терминах стационарных последовательностей}

\begin{definition} Последовательность случайных величин $\xi \equiv \{ \xi_n \}_{n \in \mathbb{Z}}$
с $\mathbb{E}|\xi_n|^2 < \infty ,~ n \in \mathbb{Z}$ называется стационарной в широком смысле,
если для всех $n \in \mathbb{Z}$
$$ \mathbb{E}\xi_n = \mathbb{E}\xi_0 ,$$
$$ R(n) \equiv \mathrm{cov}(\xi_n, \xi_0) = \mathrm{cov}(\xi_{k+n}, \xi_k) ~ \forall k \in \mathbb{Z} .$$
$R(n)$ называется ковариационной функцией. \cite{ShiryaevVeroyatnost1980} \\
\end{definition}

Рассмотрим последовательность $\xi \equiv \{ \xi_n \}$ ортонормированных случайных величин $\xi_n$:
${\mathbb{E}\xi_n = 0}$, $\mathbb{E}\xi_i\xi_j = \delta_{ij}$, где $\delta_{ij}$ -- символ Кронекера.
Данная последовательность очевидно является стационарной и
$$ R(n) = \begin{cases} 1 ,~ n = 0 \\ 0 ,~ n \neq 0 \end{cases} .$$
Функция $R(n)$ может быть представлена в виде:
$$ R(n) = \int_{-\pi}^{\pi} e^{i \lambda n} dF(\lambda) ,$$
$$ F(\lambda) = \int_{-\pi}^{\lambda} f(\nu) d\nu ,~ f(\lambda) \equiv \frac{1}{2\pi} ,~ \lambda \in [ -\pi, \pi ) .$$
Здесь $F(\lambda)$ является функцией распределения,
а $f(\lambda) \equiv \displaystyle\frac{1}{2\pi}$ -- её спектральной плотностью,
которая постоянна на непрерывном промежутке $\lambda \in [ -\pi, \pi )$.
В этом смысле можно сказать, что последовательность $\xi \equiv \{ \xi_n \}$
"составлена из гармоник $\lambda$, интенсивность которых одинакова".
Именно это обстоятельство и послужило поводом называть данную последовательность белым шумом.

