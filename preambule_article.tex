\documentclass[a4paper, 12pt]{article}
\def\MakeUppercaseUnsupportedInPdfStrings{\scshape}

%%% Работа с русским языком
\usepackage{cmap}					% поиск в PDF
\usepackage{mathtext} 				% русские буквы в формулах
\usepackage[T2A]{fontenc}			% кодировка
\usepackage[utf8]{inputenc}			% кодировка исходного текста
\usepackage[russian]{babel}	% локализация и переносы

%%% Дополнительная работа с математикой
\usepackage{amsmath,amsfonts,amssymb,amsthm,mathtools, amscd} % AMS
\usepackage{icomma} % "Умная" запятая: $0,2$ --- число, $0, 2$ --- перечисление

%% Номера формул
%\mathtoolsset{showonlyrefs=true} % Показывать номера только у тех формул, на которые есть \eqref{} в тексте.

%% Шрифты
\usepackage{euscript}	 % Шрифт Евклид
\usepackage{mathrsfs} % Красивый матшрифт

%% Поля
\usepackage[left=2cm,right=2cm,top=2cm,bottom=2cm,bindingoffset=0cm]{geometry}

%% Русские списки
\usepackage{enumitem}
\makeatletter
\AddEnumerateCounter{\asbuk}{\russian@alph}{щ}
\makeatother

%%% Работа с картинками
\usepackage{graphicx}  % Для вставки рисунков
\graphicspath{{images/}{images2/}}  % папки с картинками
\setlength\fboxsep{3pt} % Отступ рамки \fbox{} от рисунка
\setlength\fboxrule{1pt} % Толщина линий рамки \fbox{}
\usepackage{wrapfig} % Обтекание рисунков и таблиц текстом

%%% Работа с таблицами
\usepackage{array,tabularx,tabulary,booktabs} % Дополнительная работа с таблицами
\usepackage{longtable}  % Длинные таблицы
\usepackage{multirow} % Слияние строк в таблице
\usepackage[table,xcdraw]{xcolor} % Цветные таблицы

%% Красная строка
\setlength{\parindent}{2em}

%% Интервалы
\linespread{1}
\usepackage{multirow}

%% TikZ
\usepackage{tikz}
\usetikzlibrary{graphs,graphs.standard}

%% Верхний колонтитул
\usepackage{fancyhdr}
\pagestyle{fancy}

%% Перенос знаков в формулах (по Львовскому)
\newcommand*{\hm}[1]{#1\nobreak\discretionary{}
	{\hbox{$\mathsurround=0pt #1$}}{}}

%% Мои дополнения
\usepackage{booktabs} %Добавляет красивые таблицы по стандартам журналов
\usepackage{float} %Добавляет возможность работы с командой [H] которая улучшает расположение на странице
\usepackage{gensymb} %Красивые градусы
\usepackage{graphicx}               % Импорт изображений
\usepackage{caption} % Пакет для подписей к рисункам, в частности, для работы caption*
\usepackage{dsfont}
\usepackage{upgreek}    
\usepackage{wasysym}  
\usepackage{ifthen}
\usepackage[stable]{footmisc}
\usepackage{diagbox}
\usepackage{ tipa }
\usepackage{citeref}

\usepackage{hyperref}
\hypersetup{
	colorlinks,
	citecolor=black,
	filecolor=black,
	linkcolor=black,
	urlcolor=black
}

%%% Теоремы
\theoremstyle{plain}                    % Это стиль по умолчанию, его можно не переопределять.
\renewcommand\qedsymbol{$\blacksquare$} % переопределение символа завершения доказательства

\newtheorem{theorem}{Теорема}[section] % Теорема (счетчик по секиям)
\newtheorem{proposition}{Утверждение}[section] % Утверждение (счетчик по секиям)
\newtheorem{definition}{Определение}[section] % Определение (счетчик по секиям)
\newtheorem{corollary}{Следствие}[theorem] % Следстиве (счетчик по теоремам)
\newtheorem{problem}{Задача}[section] % Задача (счетчик по секиям)
\newtheorem*{remark}{Замечание} % Замечание
\newtheorem{lemma}{Лемма}[section] % Лемма (счетчик по секиям)

\newtheorem{example}{Пример}[section] % Пример
\newtheorem{counterexample}{Контрпример}[section] % Контрпример

\newtheorem{theorem_sub}{Теорема}[subsection]

\newcommand{\eqdef}{\stackrel{\mathrm{def}}{=}}
\newcommand{\ryad}{\sum\limits^{\infty}_{k = 0}}

%\newcommand*{\contfrac}[2]{%
	%{
	%	\rlap{$\dfrac{1}{\phantom{#1}}$}%
	%	\genfrac{}{}{0pt}{0}{}{#1+#2}%
	%}

\newcommand{\R}{\mathbb{R}}
\newcommand{\N}{\mathbb{N}}
\newcommand{\series}{\sum\limits_{k=1}^{\infty}}
\newcommand{\useries}{\sum\limits_{k=1}^{\infty} u_k}
\newcommand{\useriesl}{\sum\limits_{k=1}^{\infty} u_k < \infty}
\newcommand{\useriese}{\sum\limits_{k=1}^{\infty} u_k = \infty}
\newcommand{\auseries}{\sum\limits_{k=1}^{\infty} |u_k|}
\newcommand{\auseriesl}{\sum\limits_{k=1}^{\infty} |u_k| < \infty}
\newcommand{\auseriese}{\sum\limits_{k=1}^{\infty} |u_k| = \infty}
\newcommand{\sn}{\sum\limits_{k=1}^{n} u_k}
\DeclareMathOperator{\sgn}{\mathop{sgn}}

\newcommand{\te}{\ensuremath{\; \Rightarrow \;}}
\newcommand{\x}{\cdot}
\newcommand
{\un}[1]
{\ensuremath{\text{#1}}}
\newcommand{\eds}{\ensuremath{ \mathscr{E}}}
\newcommand{\ga}{\ensuremath{\gamma}}

\renewcommand {\ge}{\geqslant}
\renewcommand {\le}{\leqslant}
\renewcommand {\geq}{\geqslant}
\renewcommand {\leq}{\leqslant}
\renewcommand {\epsilon}{\varepsilon}
\newcommand{\RomanNumeralCaps}[1]
{\MakeUppercase{\romannumeral #1}}


\DeclareMathOperator{\cov}{cov}

\newenvironment{Definition}[2][]
{
	\vspace{1.2\baselineskip}
	\noindent\textbf{Определение}
	\ifthenelse{\equal{#1}{}}
	{}
	{	
		\textbf{(#1)}	
	}
	{\footnotesize [источник: #2]}
	
	\vspace{1.6mm} \small
	
}{\vspace{1.2\baselineskip}}


\newenvironment{Theorem}[2][]
{
	\vspace{1.2\baselineskip}
	\noindent\textbf{Теорема}
	\ifthenelse{\equal{#1}{}}
	{}
	{	
		\textbf{(#1)}	
	}
	{\footnotesize [источник: #2]}
	
	\vspace{1.6mm} \small
	
}{\vspace{1.2\baselineskip}}


\newenvironment{Proof}
{
	\noindent\textbf{\textit{Доказательство:}} \\
}{$\qed$\vspace{1.2\baselineskip}}


\newenvironment{Remark}[1]
{
	\vspace{1.2\baselineskip}
	\noindent\textbf{Замечание}
	{\footnotesize [источник: #1]}
	
	\vspace{1.6mm} \small
}{\vspace{1.2\baselineskip}}


\newenvironment{Proposal}[1]
{
	\vspace{1.2\baselineskip}
	\noindent\textbf{Предложение}
	{\footnotesize [источник: #1]}
	
	\vspace{1.6mm} \small
}{\vspace{1.2\baselineskip}}